\section{Нестационарное уравнение переноса}

\subsection{Двухслойные схемы для нестационарных уравнений}

\subsubsection{Определение}

Рассмотрим дифференциальное уравнение вида

\begin{equation}
    \label{eq:nonstat_common}
    \dfr{u}{t} = f, \quad f(x, t) = Lu(x, t) + g(x, t)
\end{equation}
где $L$ -- произвольный пространственный дифференциальный оператор.
При использовании двухслойной схемы аппроксимации производная по времени записывается в
виде конечной разности с шагом $\dt$, которая может приближать производную
в одном из трёх моментов времени:

\begin{equation}
    \label{eq:nonstat_dt_appr}
    \begin{array}{llll}
        \dfrac{u(t+\dt) - u(t)}{\dt} = 
            & \quad \left.\ddfr{u}{t}\right|_{t}
            & + o(\dt)
            & -\text{ разность вперёд};\\[5pt]
        \phantom{a}
            & \quad \left.\ddfr{u}{t}\right|_{t+\dt}
            & + o(\dt)
            & -\text{ разность назад};\\[5pt]
        \phantom{a}
            & \quad \left.\ddfr{u}{t}\right|_{t+\frac{\dt}{2}}
            & + o(\dt^2)
            & -\text{ симметричная разность.}
    \end{array}
\end{equation}

Момент времени $t$ будем называть текущим временн\'{ы}м слоем,
момент $t + \dt$ -- следующим,
а момент $t+\dt/2$ -- промежуточным.
Считается, что
значение функции на текущий момент времени $u(t)$ известно, а
значение на следующий момент $u(t+\dt)$ подлежит определению.

\subsubsubsection{Явная схема}

При использовании разности назад уравнение \eqref{eq:nonstat_common}
в полудискретизованном (то есть дискретизованном только по времени, но не по пространству) виде
запишется как
\begin{equation*}
    \frac{u(x, t+\dt) - u(x)}{\dt} - L u(x, t) = g(x, t)
\end{equation*}
или, после переноса всех известных слагаемых вправо
\begin{equation}
    \label{eq:nonstat_explicit}
    u(x, t+\dt) = \left(E + \dt L\right) u(x, t) + \dt g(x, t).
\end{equation}
Здесь $E$ -- единичный оператор.
Схема \eqref{eq:nonstat_explicit} называется явной схемой
и имеет первый порядок точности.

\subsubsubsection{Неявная схема}

Выбрав разность назад из выражения \eqref{eq:nonstat_dt_appr}
полудискретизованная схема для уравнения \eqref{eq:nonstat_common} примет вид
\begin{equation*}
    \frac{u(x, t+\dt) - u(x)}{\dt} - L u(x, t+\dt) = g(x, t+\dt).
\end{equation*}
В результате преобразования получим неявную схему первого порядка точности
\begin{equation}
    \label{eq:nonstat_implicit}
    \left(E-\dt L\right) u(x, t+\dt) = u(x, t) + \dt g(x, t + \dt).
\end{equation}

\subsubsubsection{Схема Кранка--Николсон}

Подставим симметричную разность из \eqref{eq:nonstat_dt_appr}
в уравнение \eqref{eq:nonstat_common}. Формально получим
\begin{equation*}
    \frac{u(x, t+\dt) - u(x)}{\dt} - L u(x, t+\frac{\dt}{2}) = g(x, t+\frac{\dt}{2}).
\end{equation*}

Для определения выражения функций на промежуточном временном слое
распишем значение $u$ на текущем и следующем слоях в ряд Тейлора
относительно значения на момент $t+\dt/2$:
\begin{align*}
    u\left(t\right)      &= u\left(t+\frac{\dt}{2}\right) - \left.\frac{\dt}{2}\dfr{u}{t}\right|_{t+\frac{\dt}{2}} + o(\dt^2)\\
    u\left(t+\dt\right) &= u\left(t+\frac{\dt}{2}\right) + \left.\frac{\dt}{2}\dfr{u}{t}\right|_{t+\frac{\dt}{2}} + o(\dt^2)
\end{align*}
Взяв полусумму этих выражений получим аппроксимацию функции на промежуточном слое:
\begin{equation}
    \label{eq:nonstat_cn_appr}
    u\left(x, t + \frac{\dt}{2}\right) = \frac12u\left(x, t\right) + \frac12u\left(x, t+\dt\right) + o(\dt^2)
\end{equation}
Аналогичная запись справедлива и для свободного члена $g$.
Если оператор $L$ -- нестационарный или нелининый, то аппроксимацию \eqref{eq:nonstat_cn_appr}
следует записывать для всего выражения $Lu$:
\begin{equation*}
    \left(Lu\right)_{t+\frac{\dt}{2}} = \frac12\left(Lu\right)_{t} + \frac12\left(Lu\right)_{t+\dt} + o(\dt^2)
\end{equation*}


С учётом \eqref{eq:nonstat_cn_appr} симметричная разностная схема
запишется как
\begin{equation*}
    \frac{u(x, t+\dt) - u(x)}{\dt} - \frac12L u(x, t) - \frac12 L u(x, t + \dt) = \frac12 g(x, t) + \frac12 g(x, t+\dt)
\end{equation*}
или
\begin{equation}
    \label{eq:nonstat_cn}
    \left(E-\frac{\dt}{2} L\right) u(x, t+\dt) = \left(E + \frac{\dt}{2} L\right) u(x, t) + \frac{\dt}{2} \left(g(x, t) + g(x, t + \dt)\right).
\end{equation}
Такая схема называется схемой Кранка--Николсон и имеет второй порядок аппроксимации по времени.

В случае, если оператор $L$ зависит от времени, то
в левой части схемы \eqref{eq:nonstat_cn} его нужно
брать на следующем временном слое, а в правой -- на текущем.

\subsubsubsection{Обобщённая двухслойная схема}

Выражения
\eqref{eq:nonstat_explicit},
\eqref{eq:nonstat_implicit},
\eqref{eq:nonstat_cn}
можно записать в обобщённой форме
\begin{equation}
    \label{eq:nonstat_theta}
    \left(E-\theta\dt L\right) u(x, t+\dt) = \left(E + \left(1 - \theta\right)\dt L\right) u(x, t) + \left(1 - \theta\right) g(x, t) + \theta g(x, t + \dt).
\end{equation}
Коэффициент $\theta$ -- степень неявности схемы:
\begin{itemize}
\item $\theta = 0$ -- явная схема \eqref{eq:nonstat_explicit},
\item $\theta = 1$ -- полностью неявная схема \eqref{eq:nonstat_implicit},
\item $\theta = 1/2$ -- схема Кранка--Николсон \eqref{eq:nonstat_cn}.
\end{itemize}

Отметим, что только при $\theta = 1/2$ схема \eqref{eq:nonstat_theta} имеет второй порядок точности по времени.
Для других значений (в том числе промежуточных) схема будет иметь ошибку первого порядка $o(\dt)$.

\subsection{Схемы высокого порядка точности}
Существуют два подхода к построению схем дискретизации по времени произвольного высокого порядка:
первый из них предполагает использование нескольких временных слоев: $t+\dt, t, t-\dt, t-2\dt, ...$, второй -- использование
большого количества промежуточных точек на единственном временном отрезке: $t+c_i\dt, c_i\in[0, 1]$.
Первый подход используется для построения схем Адамса, второй -- схем Рунге--Кутта.

\subsubsection{Многослойные схемы. Схемы Адамса}
В общем виде многослойную расчётную схему можно записать в виде
\begin{equation}
\label{eq:nonstat_multilayer}
\sum\limits_{i=1}^{s} a_i u_i = \dt \sum\limits_{i=1}^{s} b_i f_i.
\end{equation}
Здесь нижние индексы функций использованы для указания временного слоя:
\begin{equation*}
\begin{aligned}
&u_i = u(x, t + (i+1-s)\dt),\\
&f_i = g(x, t + (i+1-s)\dt) - Lu_i,
\end{aligned}
\end{equation*}
а $s$ -- это общее количество используемых временных слоёв.
Значение всех функций на слоях $i<s$ считаются известными, а значение $u_s$ на последнем (текущем) слое подлежит определению.
Коэффициенты $a_i$. $b_i$ определяют конкретную схему.
Для однозначности принято задавать коэфиициет перед значением $u$ на текущем слое равным единице: $a_s = 1$.
Так, для двухслойной ($s=2$) схемы Кранка--Николсон \cref{eq:nonstat_cn} эти коэффициенты примут вид:
\begin{align*}
&a_1 = -1, \quad a_2 = 1,\\
&b_1 = 1/2, \quad b_2 = 1/2.
\end{align*}
Для того чтобы схема, записанная в общем виде \cref{eq:nonstat_multilayer}, была явной,
необходимо, чтобы выполнялось условие $b_s = 0$.

\subsubsubsection{Явные схемы Адамса--Башфорта}
Запишем интерполяционный полином для правой части уравнения \cref{eq:nonstat_common}
по значениям на предыдущих временных слоях $f_i, i<s$:
\begin{equation*}
p(t) = \sum\limits_{i=1}^{s-1}\left(f_i \prod\limits_{\substack{m=1\\m \neq i}}^{s-1} \frac{t - t_i}{t_m - t_i} \right).
\end{equation*}
Порядок аппроксимации этого полинома на единицу больше его степени. С учётом 
$t_{i+1} - t_{i} = \dt$ можно записать
\begin{equation*}
f(t) = p(t) + o(\dt^{s-1}).
\end{equation*}

Далее проинтегрируем уравнение по текущему временному отрезку: $[t_{s-1}, t_{s}]$:
\begin{equation*}
u_s - u_{s-1} = \sum_{i=1}^{s-1}\left( f_i \int\limits_{t_{s-1}}^{t_s} \prod \limits_{\substack{m=0\\m\neq i}}^{s-1} \frac{t - t_i}{t_m - t_i}\, dt \right).
\end{equation*}

Отсюда коэффициенты в общей форме \cref{eq:nonstat_multilayer} примут вид
\begin{equation*}
a_i = 
\begin{cases}
1,  &\quad i = s\\
-1, &\quad i = s-1\\
0,  &\quad i < s-1
\end{cases}
\qquad
b_i =
\begin{cases}
0, &\quad i = s\\[10pt]
\displaystyle\frac{1}{\dt}\int\limits_{t_{s-1}}^{t_s} \prod \limits_{\substack{m=0\\m\neq i}}^{s-1} \frac{t - t_i}{t_m - t_i}\, dt, &\quad i < s.
\end{cases}
\end{equation*}

Примеры трёх-, четырех- и пятислойной явных схем приведены ниже:
\begin{align*}
&\frac{u_3 - u_2}{\dt} = \frac{3f_2 - f_1}{2} + o(\dt^2),\\
&\frac{u_4 - u_3}{\dt} = \frac{23 f_3 - 16 f_2 + 5 f_1}{12} + o(\dt^3),\\
&\frac{u_5 - u_4}{\dt} = \frac{55 f_4 - 59 f_3 + 37 f_2 - 9 f_1}{24} + o(\dt^4).
\end{align*}

\subsubsubsection{Неявные схемы Адамса--Мультона}
Теперь снимем требование явности и запишем интерполяционный полином для правой части исходного уравнения \cref{eq:nonstat_common} с использованием $s$ точек:
\begin{equation*}
p(t) = \sum\limits_{i=1}^{s}\left(f_i \prod\limits_{\substack{m=1\\m \neq i}}^{s} \frac{t - t_i}{t_m - t_i} \right).
\end{equation*}
Тогда порядок аппроксимации этого полинома будем на единицу больше, чем для полинома явного метода.
Далее, проинтегрируем исходное уравенение по отрезку $t_{s-1}, t_s$ и получим коэффициенты общей формы \cref{eq:nonstat_multilayer}:
\begin{equation*}
a_i = 
\begin{cases}
1,  &\quad i = s\\
-1, &\quad i = s-1\\
0,  &\quad i < s-1
\end{cases}
\qquad
b_i =
\displaystyle\frac{1}{\dt}\int\limits_{t_{s-1}}^{t_s} \prod \limits_{\substack{m=0\\m\neq i}}^{s} \frac{t - t_i}{t_m - t_i}\, dt.
\end{equation*}

Двуслойная схема Адама--Мультона будет аналогична неявной двухслойной схеме \cref{eq:nonstat_implicit},
трёхслойная -- схеме Кранка--Николсон \cref{eq:nonstat_cn}. Примеры схемы высоких порядков представлены ниже:
\begin{align*}
&\frac{u_3 - u_2}{\dt} = \frac{5f_3 +  8 f_2 - f_1}{12} + o(\dt^3),\\
&\frac{u_4 - u_3}{\dt} = \frac{9 f_4 + 19 f_3 - 5 f_2 + f_1}{24} + o(\dt^4),\\
&\frac{u_5 - u_4}{\dt} = \frac{251 f_5 + 646 f_4 - 264 f_3 + 106 f_2 - 19 f_1}{720} + o(\dt^5).
\end{align*}

\subsubsection{Схемы Рунге-Кутта}
TODO

\subsection{Устойчивость расчётных схем}
\label{sec:stability_analysis}

\subsubsection{Дискретизация по времени как итерационный процесс}

\subsubsubsection{Двухслойный итерационный процесс}

Простой двухслойный итерационный процесс определяется как
\begin{equation}
    \label{eq:nonstat_common_iter}
    u^{n+1} = A u^{n} + b,
\end{equation}
где $n$ -- индекс итерационного слоя,
$A$ -- оператор преобразования,
$b$ -- свободный член.

Определение значения функции на следующий момент времени $u(t+\dt)$ 
по двухслойной схеме \eqref{eq:nonstat_theta} можно представить как простой итерационный процесс \eqref{eq:nonstat_common_iter}, где
\begin{align*}
    A &= \left(E - \theta \dt L \right)^{-1} \left(E + (1 - \theta) \dt L \right), \\\\
    b &= \left(E - \theta \dt L \right)^{-1} \left( \theta g\left(x, t + \dt\right) + \left(1 - \theta\right) g\left(x, t\right) \right).
\end{align*}

Итерационный процесс называется сходящимся, если
\begin{equation*}
    \lim_{n=\infty} \left\lVert u^{n+1} - u^{n} \right\rVert = 0.
\end{equation*}

\subsubsubsection{Устойчивость итерационного процесса}
\label{sec:ScalarIter}

Рассмотрим два простых итерационных процесса,
имеющих на нулевом слое значение $u^0 = 1$:
\begin{align*}
    {\rm (I)}:\quad  &u^{n+1} = 2 u^{n} - 1, \\
    {\rm (II)} :\quad  &u^{n+1} = 0.5 u^{n} + 0.5.
\end{align*}
Оба этих процесса при выбранном начальном приближении, очевидно, сходятся.
На каждой итерации справделиво $u^n = 1$.
Возмутим начальное условие: пусть
\begin{equation*}
    u^0 = 1 + \eps,
\end{equation*}
и проведём итерации.
\begin{equation*}
    \def\arraystretch{1.7}
    \begin{array}{c|c|c}
        & {\rm (I)}  &  {\rm (II)} \\
        \hline
        u^1 & 1 + 2\eps & 1 + \dfrac{\eps}{2} \\
        \hline
        u^2 & 1 + 4\eps & 1 + \dfrac{\eps}{4} \\
        \hline
        u^3 & 1 + 8\eps & 1 + \dfrac{\eps}{8} \\
        \hline
        ... & & \\
        \hline
        u^\infty & \infty & 1 \\
    \end{array}
\end{equation*}

Видно, что процесс $\rm (I)$ теряет сходимость и стремится к бесконечности,
в то время, как процесс $\rm (II)$ сохраняет свои свойства.

Свойство итерационных процессов уменьшать малые возмущения называется устойчивостью.
В примере выше процесс $\rm (I)$ является неустойчивым, а процесс $\rm (II)$ -- устойчивым.

Нетрудно видеть, что для рассматриваемого скалярного итерационного процесса,
условие устойчивости запишется в виде $|A| \leq 1$.

\subsubsubsection{Источники возмущений}

На практике возникновение возмущений в решениях неизбежно:
они могут быть следствием ошибок дискретизации функций и операторов,
погрешностей решения СЛАУ, ошибок при проведении арифметических
операций на числах с плавающей точкой и т.д.
Поэтому любой итерационный процесс, используемый для решений
математических задач, должен быть устойчив.

Возникновение непреднамеренных ошибок вследствии компьютерного округления можно проиллюстрировать на примере
программы, в которой рассматривается сходящийся для любого начального условия, но неустойчивый итерационный процесс
\begin{equation*}
    u^{n+1} = 10 u^{n} - 9 u^{0}.
\end{equation*}

\begin{minted}[linenos=false]{c++}
double u0 = 0.625;
double u = u0;
for (int i=0; i<1000; ++i){
    u = 10*u - 9*u0;
}
std::cout << u << std::endl;
\end{minted}

Если начальное значение может быть точно представлено в
числах с плавающей точкой (путём конечной суммы степеней двойки),
то арифметическая ошибка не возникает.
Так, представленный выше код на выходе печатает ожидаемое $u=0.625$.
Потому что начальное приближение может быть разложено как $u^0 = 2^{-1} + 2^{-3}$.

Однако, если заменить начальное приближение на любое число,
которое не может быть записано точно во floating-point формате,
то процесс быстро уходит в бесконечность.
Например, для $u^0 = 0.626$ бесконечные (непредставимые в машинном формате) значения
появляются на 324-ой итерации,
а при переключении на работу в числах одинарной точности `float` -- уже на 46-ой.

\subsection{Свойства двухслойной расчётной схемы}
TODO

\subsection{Аппроксимация уравнения переноса с ограничением потока}

\subsubsection{Схемы первого и второго порядка точности}
Рассмотрим уравнение переноса в одномерной постановке

\begin{equation*}
\label{eq:tvd_transport}
\dfr{u}{t} + U\dfr{u}{x} = 0
\end{equation*}
Все дальнейшие выкладки будем приводить исходя из условия положительности скорости переноса $U > 0$.

Рассмотрим два вида пространственной аппроксимации конвективного слагаемого на равномерной сетке: схемой против потока и симметричной схемой
\begin{align}
& \dfr{u_i}{t} + U \frac{u_i - u_{i-1}}{h} = 0, \label{eq:tvd_upwind}\\
& \dfr{u_i}{t} + U \frac{u_{i+1} - u_{i-1}}{2h} = 0. \label{eq:tvd_sym}
\end{align}
Первая схема является (условно) устойчивой но при этом обладает первым
порядком аппроксимации. Вторая неустойчива, но имеет порядок $o(h^2)$.
Идея методов аппроксимации с ограничением потока состоит в том,
чтобы на основе комбинации первой и второй схем
построить устойчивое решение, имеющее ``почти везде'' второй порядок аппроксимации.

Запишем эти аппроксимации в общем виде:
\begin{equation}
\label{eq:tvd_appr_with_f}
\dfr{u_i}{t} = \frac{f_{i-\sfrac12} - f_{i+\sfrac12}}{h}.
\end{equation}
Здесь $f_{i+\sfrac12}$ -- численный поток, который в зависимости от выбранной схемы будет равен
\begin{equation}
\label{eq:tvd_fhl}
\begin{aligned}
&f^L_{i+\sfrac12} = U u_i                            &\text{-- схема против потока}\\
&f^H_{i+\sfrac12} = U \frac{u_i + u_{i+1}}{2}        &\text{-- симметричная схема}.
\end{aligned}
\end{equation}
Здесь $f^L$, $f^H$ означают потоки низкого (Low) и высокого (High) порядка аппроксимации.

Аппроксимацию с ограничением потока запишем в виде
\begin{equation}
\label{eq:tvd_f}
f_{i+\sfrac12} = f^L_{i+\sfrac12} + \Phi_{i+\sfrac12} \left( f^H_{i+\sfrac12} - f^L_{i+\sfrac12} \right).
\end{equation}
$\Phi$ в этой записи называется ограничителем, который служит
переключателем: при $\Phi = 0$ мы получаем схему первого порядка, при $\Phi=1$ -- схему второго порядка.

Далее будем выбирать $\Phi$ таким образом, чтобы не допустить возникновения осцилляций в численном решении.

\subsubsection{Условие TVD}
В качестве критерия, характеризующего возникновение и развитие осцилляций, выберем полную вариацию:
\begin{equation}
\label{eq:tvd_tv}
\begin{WithArrows}
TV(u) =& \displaystyle\int\left| \nabla u \right|\,dx =           \Arrow{в одномерном случае}\\[10pt]
      =& \displaystyle\int\left| \dfr{u}{x} \right| \, dx =       \Arrow{для сеточной функции}\\[10pt]
      =& \displaystyle\sum_{i}\left| u_i - u_{i-1}\right|.
\end{WithArrows}
\end{equation}
Условие уменьшения осцилляций в решении на следующем временном слое примет вид
\begin{equation*}
TV(\hat u) \leq TV(u).
\end{equation*}
Численные схемы, удовлетворяющие этому условию, называются TVD (Total variation diminishing) схемами.

Запишем численную схему в общем виде
\begin{equation}
\label{eq:tvd_harten}
\dfr{u_i}{t} = c_{i-\sfrac12} (u_{i-1} - u_{i}) + c_{i+\sfrac12} (u_{i+1} - u_{i}).
\end{equation}
Согласно теореме Хартена такая схема удовлетворяет свойству TVD, если $c_{i\pm\sfrac12} \geq 0$.
Схема против потока \cref{eq:tvd_upwind} является TVD-схемой:
$$c_{i-\sfrac12} = \frac{U}{h}, \quad c_{i+\sfrac12} = 0,$$
а симметричная схема \cref{eq:tvd_sym} -- нет:
$$c_{i-\sfrac12} = \frac{U}{2h}, \quad c_{i+\sfrac12} = -\frac{U}{2h}.$$

Подставляя уравнение \cref{eq:tvd_f} в \cref{eq:tvd_appr_with_f} 
и приводя к форме \cref{eq:tvd_harten} получим
\begin{equation}
\label{eq:tvd_phi_condition1}
\dfr{u_i}{t} = \frac{U}{2h}\left( 2 - \Phi_{i-\sfrac12} \right) \left( u_{i-1} - u_{i} \right)
               +\frac{U}{2h}\left(-\Phi_{i+\sfrac12} \right) \left( u_{i+1} - u_i\right)
\end{equation}
То есть для удовлетворения свойства TVD необходимо, чтобы $\Phi \leq 0$.
Для второго порядка точности требуется $\Phi = 1$. То есть линейные схемы TVD не могут иметь высокий порядок точности.

\subsubsection{Нелинейные TVD схемы}
Для того, чтобы преодолеть это ограничение, будем строить нелинейные схемы.
Общая идея построения таких схем состоит в том, чтобы выбрать такую $\Phi$,
при которой второе слагаемое равенства \cref{eq:tvd_phi_condition1} можно
было отнести к первому. То есть можно бы было записать
\begin{equation}
\label{eq:tvd_phi_condition2}
\Phi_{i+\sfrac12}(u_{i+1} - u_{i}) = -\Phi'_{i+\sfrac12} (u_{i-1} - u_{i}).
\end{equation}
Тогда условием TVD станет выражение
\begin{equation}
\label{eq:tvd_phi_condition3}
2 - \Phi_{i-\sfrac12} + \Phi'_{i+\sfrac12} \geq 0.
\end{equation}

Для характеристики поведения функции выберем соотношение наклонов (slope ratio),
который в одномерном виде запишется в виде:
\begin{equation}
\label{eq:tvd_ri}
r_i = \frac{u_i - u_{i-1}}{u_{i+1} - u_{i}}
\end{equation}
Нелинейность схемы будет выражаться в зависимости
\begin{equation*}
\Phi_{i+\sfrac12} = \Phi(r_i).
\end{equation*}

Из \cref{eq:tvd_phi_condition2} следует
\begin{equation*}
\Phi'_{i+\sfrac12} = \frac{\Phi(r_i)}{r_i}
\end{equation*}
а неравенство $\label{eq:tvd_phi_condition4}$
перепишется в виде
\begin{equation*}
\label{eq:tvd_phi_condition5}
2 - \Phi(r_i) + \frac{\Phi(r_i)}{r_i} \geq 0.
\end{equation*}
Чтобы из этого условия получить ограничение для $\Phi$, явно не зависящее от $r_i$, потребуем
\begin{equation}
\label{eq:tvd_phi_condition4}
\Phi\left(\frac{1}{r_i}\right) = \frac{\Phi(r_i)}{r_i}.
\end{equation}
Тогда неравентсво \cref{eq:tvd_phi_condition3} примет вид
\begin{equation*}
\label{eq:tvd_phi_condition6}
2 - \Phi(r_i) + \Phi\left(\frac{1}{r_i}\right) \geq 0.
\end{equation*}
Отсюда получим условие для $\Phi$:
\begin{equation}
\label{eq:tvd_phi_condition7}
0 \leq \Phi(r_i) \leq 2.
\end{equation}

Дополнительно потребуем, чтобы в точках с гладким поведением функции использовать схему второго порядка точности:
\begin{equation}
\label{eq:tvd_phi_condition8}
\Phi(1) = 1
\end{equation}
а в точках локального экстремума (которые особенно подвержены появлению осцилляций) гарантировать переключение на схему первого порядка:
\begin{equation}
\label{eq:tvd_phi_condition9}
\Phi(r\leq0) = 0.
\end{equation}

Таким образом, для построения TVD схемы, функция ограничитель должна удовлетворять условиям
\cref{eq:tvd_phi_condition4,eq:tvd_phi_condition7,eq:tvd_phi_condition8,eq:tvd_phi_condition9}.
Ниже представлены некоторые часто используемые ограничители, удовлетворяющие этим свойствам:
\begin{equation}
\label{eq:tvd_limiter}
\Phi(r) = \left\{
\begin{array}{ll}
\max(0, \min(r, 1))                   & \text{-- minmod};\\[10pt]
\dfrac{r + |r|}{1+|r|}                & \text{-- Van Leer}; \\[10pt]
\max(0, \min(2r, \dfrac{1+r}{2}, 2)   & \text{-- monotonized central (MC)}; \\[10pt]
\max(0, \min(2, r), \min(1, 2 r))     & \text{-- superbee}.
\end{array}
\right.
\end{equation}

\subsection{TVD-схемы для неструктурированных конечнообъёмных сеток}
\label{sec:tvd_fvm}
Рассмотрим многомерное уравнение переноса
\begin{equation}
\nonumber
\dfr{u}{t} + \vec U \cdot \nabla u = 0.
\end{equation}

Применим конечнообъёмную процедуру
для получения слабой интегральной постановки задачи.
Для этого проинтегрируем это уравнение
по конечному объёму $E_i$ 
и применим формулу интегрирования по частям.
Получим
\begin{equation}
\label{eq:tvd_fvm_transport}
\left| V_i \right|
\dfr{u}{t}
+ \sum_{j\in {\rm nei}(i)} {f_{ij} \left|\gamma_{ij}\right|}
= 0, \quad f_{ij} = u_{ij} U_{ij}.
\end{equation}
Здесь $|V_i|$ -- объём конечного элемента,
${\rm nei}(i)$ -- совокупность
всех точек коллокации, инцидентных ячейке $i$
(центров соседних ячеек и соседних граничных граней),
$f_{ij}$ -- поток из точки коллокации $i$ в точку коллокации $j$,
$|\gamma_{ij}|$ --
площадь грани
конечного объёма $i$, через
которую этот объём соединяется с точкой коллокации $j$,
$u_{ij}$ -- значение функции $u$, отнесённое к этой грани,
$U_{ij}$ -- скорость потока в направлении внешней по отношению
к ячейке $i$ нормали.

Для потока справедливо
\begin{equation}
\label{eq:tvd_fij_fji}
f_{ij} = -f_{ji}.
\end{equation}
То есть для вычисления потока на грани достаточно найти значение для одного направления.
Выберем это направление $\overrightarrow{ij}$ таким образом, чтобы $U_{ij} > 0$ (рис.~\ref{fig:fvm_tvd}).


\begin{figure}[h!]
\centering
\includegraphics[width=0.4\linewidth]{fvm_tvd.pdf}
\caption{Вспомогательный узел $\vec c^p_{ij}$ на конечнообъёмной сетке}
\label{fig:fvm_tvd}
\end{figure}

Будем считать, что скорость переноса $U$ -- известная функция.
Тогда запишем значения потоков высокого и низкого порядка согласно \cref{eq:tvd_fhl}:
\begin{equation}
\label{eq:tvd_fhl_fvm}
\begin{aligned}
&f^L_{ij} = U_{ij} u_i                 &\text{-- схема против потока}\\
&f^H_{ij} = U_{ij} \frac{u_i + u_j}{2} &\text{-- симметричная схема}.
\end{aligned}
\end{equation}

Поток при этом запишется по аналогии с \cref{eq:tvd_f}:
\begin{equation}
\label{eq:tvd_f_fvm}
f_{ij} = f^L_{ij} + \Phi(r_{ij}) \left( f^H_{ij} - f^L_{ij} \right).
\end{equation}

В одномерном случае для записи соотношения наклонов $r_i$ \cref{eq:tvd_ri} использовались
три точки: текущий узел $i$, узел против потока $i-1$, и узел по потоку $i+1$.
Для случае неструктурированной сетки лишь две из этих трёх точек являются
узлами коллокации: текущий узел $i$ и узел по потоку $j$. Определим точку
против потока симметричным отражением: $\vec{c}^p_{ij} = 2\vec c_i - \vec c_j$ (см. рис.~\ref{fig:fvm_tvd}).
Значение функции в этой точке обозначим как $u^p_{ij}$.
Тогда соотношение наклонов запишется в виде

\begin{equation}
\label{eq:tvd_ri_fvm}
r_i = \frac{u_i - u_{ij}^p}{u_j - u_i}
\end{equation}

Точка $\vec c^p$ (в отличии от $x_{i-1}$ из одномерного случая)
не является точкой коллокации. То есть
значение $u^p$ нельзя достать из вектора столбца сеточной функции $u$.
Однако, это значение можно интерполировать
по значениям в ближайших точках коллокации.

\subsubsection{Прямая интерполяция противопоточного значения}
\label{sec:direct_interpolate_up}
Так, в двумерном случае для определения $u^p_{ij}$
необходимо найти три точки коллокации,
ближайшие к точке $\vec c^p_{ij}$ и не лежащие на одной прямой (или две точки коллокации, помимо $c_i$). На рис.~\ref{fig:fvm_tvd}
они помечены индексами $\vec c_m$, $\vec c_n$).
И далее в треуольнике, образованном этими тремя
точками ($\triangle_{imn}$) провести
интерполяцию
по формуле \cref{eq:simplex_interp_2d}.
Специально отметим, что точка $\vec c^p_{ij}$
не обязана содержаться внутри 
треугольника $\triangle_{imn}$.

\subsubsection{Интерполяция противопоточного значения через значение градиентов}
\label{sec:grad_inteprolate_up}.
Другой подход к определению $u^p_{ij}$ основан на записи симметричной конечной разности
по направлению $\vec c_{ij}$:
\begin{equation*}
\left.\dfr{u}{c_{ij}}\right|_i = \frac{u_j - u^p_{ij}}{2|\vec c_{ij}|} \qquad \hence
u_{ij}^p = u_j - 2 |\vec c_{ij}| \left. \dfr{u}{c_{ij}}\right|_i.
\end{equation*}
Производная по направлению $c_{ij}$ находится как проекция градиента:
\begin{equation*}
|\vec c_{ij}| \left.\dfr{u}{c_{ij}}\right|_i = \vec c_{ij} \cdot (\nabla u)_i.
\end{equation*}
Таким образом, задача интерполяции сводится к задаче определения градиента функции
$u$ в узлах коллокации. Эта задача была рассмотрена ранее в \secref{sec:gradu_in_cells}.

\subsubsection{Реализация для явной схемы}
Для примера рассмотрим написание TVD-схемы
рассмотрим чисто явную схему для полудискретизованного уравнения \cref{eq:tvd_fvm_transport}:
\begin{equation}
\label{eq:tvd_fvm_explicit_transport}
\left| V_i \right|
\frac{\hat u - u}{\dt}
+ \sum_{j\in {\rm nei}(i)} {f_{ij} \left|\gamma_{ij}\right|}
= 0.
\end{equation}
Для того, чтобы избежать повторного вычисления потоков $f_{ij}$ и $f_{ji}$,
будем собирать эту схему в цикле по граням. Пусть через границу притока нет (то есть для граничные граней $f_{ij} = 0$.
Тогда останется только цикл по внутренним граням:
\begin{equation}
\label{eq:tvd_fvm_assem}
\begin{array}{ll}
\hat u = u                                               & \textrm{-- инициализируем следующий шаг}\\
\textbf{for } s \in\textrm{internal}                     & \textrm{-- цикл по внутренним граням}\\ 
\qquad i,j = \textrm{nei\_cells(s)}                      & \textrm{-- две ячейки, соседние с текущей гранью}\\
\qquad \vec U_{ij}                                       & \textrm{-- вектор скорости в центре грани}\\
\qquad \vec n_{ij}                                       & \textrm{-- вектор нормали к грани от ячейки i к j}\\
\qquad U_{ij} = \vec U_{ij} \cdot \vec n_{ij}            & \textrm{-- проекция скорости на нормаль}\\
\qquad \textbf{if } U_{ij} \leq 0                        & \textrm{-- схема против потока}\\
\qquad \qquad \textrm{swap}(i, j); U_{ij} = -U_{ij}      & \textrm{-- гарантируем, что жидкость течет от $i$ к $j$}\\
\qquad \textbf{endif}                                    & \textrm{}\\
\qquad f^L_{ij} = U_{ij} u_i                             & \textrm{-- поток 1-го порядка \cref{eq:tvd_fhl_fvm}}\\
\qquad f^H_{ij} = U_{ij} (u_i + u_j) / 2                 & \textrm{-- поток 2-го порядка \cref{eq:tvd_fhl_fvm}}\\
\qquad \vec c_i, \vec c_j                                & \textrm{-- центры ячеек}\\
\qquad \vec c^p = 2 \vec c_i - \vec c_j                  & \textrm{-- вспомогательная точка}\\
\qquad u^p = \textrm{interpolate}(i, j, u, \vec c^p)     & \textrm{-- интерполируем $u$ в точке $\vec c^p$}\\
\qquad r = \sfrac{(u_i - u^p)}{(u_j - u_i)}              & \textrm{-- отношение наклонов \cref{eq:tvd_ri_fvm}}\\
\qquad F = \textrm{limiter}(r)                           & \textrm{-- ограничитель \cref{eq:tvd_limiter}}\\
\qquad f_{ij} = f^L_{ij} + F (f^H_{ij} - f^L_{ij})       & \textrm{-- вычисление потока \cref{eq:tvd_f_fvm}}\\
\qquad \hat u_i\minuseq\dt/|V_i|\,f_{ij}\,|\gamma_{ij}| & \textrm{-- добавление в противопотоковую ячейку}\\
\qquad \hat u_j\pluseq \dt/|V_j|\,f_{ij}\,|\gamma_{ij}| & \textrm{-- добавление в попотоковую ячейку}\\
\textbf{endfor}                                          & \\
\end{array}
\end{equation}
Отметим, что использование
противоположенного знака при добавлении в правую
от грани ячейку $j$ связано с тожедством \cref{eq:tvd_fij_fji}.
То есть на самом деле в ячейку $j$
должен бы добавляться поток $f_{ji}$,
но поскольку отдельной обработки этого направления не предусмотрено,
мы добавляем $f_{ij}$ с обратным знаком.
При реализации функции $\textrm{interpolate}$ должен использоваться один из методов, изложенных
в пп.~\ref{sec:direct_interpolate_up}, \ref{sec:grad_inteprolate_up}.

\subsection{Неявные нелинейные TVD-схемы}
Запишем полудискретизованное выражение \cref{eq:tvd_fvm_transport}
в матричном виде
\begin{equation*}
\mat E \dfr{u}{t} = \mat K u,
\end{equation*}
где $\mat E$ -- диагональная матрица, а $\mat K$ -- оператор переноса.
Разделим последний на линейную ($\mat L$, противопотоковую) и нелинейную ($\mat F^a$, антидиффузную) части :
\begin{equation*}
\mat K = \mat L + \mat F^a(u).
\end{equation*}
Согласно представленной в \secref{sec:tvd_fvm} схеме ненулевые элементы этих матриц вычислятся 
для каждой направленной инцидентной пары $\overrightarrow{ij}$ (при $U_{ij} \geq  0$) следующим образом
\begin{align*}
&l_{ij} = 0, \quad l_{ji} = U_{ij} |\gamma_{ij}|, \quad l_{ii} = -\sum_{j\neq i} l_{ij}\\
&f^a_{ij} = f^a_{ji} = -\tfrac12 U_{ij} \Phi_{ij} |\gamma_{ij}|, \quad f^a_{ii} = -\sum_{j\neq i} f^a_{ij}.
\end{align*}

Используя обобщённую схему \cref{eq:nonstat_theta}
нелинейная система уравнений запишется в виде
\begin{equation}
\label{eq:tvd_theta_sle}
\left(\mat E - \theta \dt L - \theta \dt F^a(\hat u)\right) \hat u = \left(\mat E + \dt (1 - \theta) L + \dt (1 - \theta) F^a(u)\right) u.
\end{equation}
На каждом временном слое такая нелинейная система может быть решена методом коррекции поправки \secref{sec:it_def_corr},
в котором в качестве предобуславливателя $\mat B$ следует взять линейную часть оператора переноса:
\begin{align*}
&\mat B = \mat E - \theta \dt L \\
&\mat A(\hat u) = \mat B - \theta \dt F^a(\hat u), \\
&f = \left(\mat E + \dt (1 - \theta) L + \dt (1 - \theta) F^a(u)\right)
\end{align*}
