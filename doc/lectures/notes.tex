\documentclass[a4paper,12pt]{article}
\usepackage[utf8]{inputenc}
\usepackage{graphicx}
\usepackage{float}%"Плавающие" картинки
\usepackage{wrapfig}%Обтекание фигур (таблиц, картинок и прочего)
\usepackage{amsmath, amssymb, amsthm}
\usepackage[T1, T2A]{fontenc}
\usepackage[english, russian]{babel}
\usepackage[left=1cm,right=1cm, top=1cm, bottom=2cm]{geometry}
% Межстрочный интервал = 1.5pt
\usepackage{setspace}
\onehalfspacing
% % Абзацный отступ = 1.25см
%\usepackage{indentfirst}
%\setlength\parindent{0mm}
% Путь до папки с изображениями
\graphicspath{ {../img/} }
% listings
\usepackage{minted}
% Активные ссылки на формулы и лит-ру
\usepackage[
  colorlinks=true,
  citecolor=blue,
  linkcolor=blue,
  linktoc=page,
]{hyperref}
\usepackage[all]{hypcap}
\usepackage{makecell}

% cleverref
\usepackage{cleveref}
\newcommand{\crefrangeconjunction}{~--~}
\newcommand{\crefpairconjunction}{, }
\newcommand{\creflastconjunction}{, }
\crefname{equation}{}{}

% enumeration
\numberwithin{equation}{section}

\usepackage{xcolor}
\usepackage{mdframed}
\usepackage{cancel}
\usepackage{tcolorbox}
\newcommand{\Ren}{\mathrm{Re}}
\newcommand{\Pen}{\mathrm{Pe}}
\newcommand{\Nun}{\mathrm{Nu}}
\renewcommand{\vec}[1]{\boldsymbol{\rm #1}}
\newcommand{\tensor}[1]{\boldsymbol{\overline{\rm #1}}}
\newcommand{\dfr}[2]{\frac{\partial #1}{\partial #2}}
\newcommand{\dfrq}[2]{\frac{\partial^2 #1}{\partial #2^2}}
\newcommand{\ddfr}[2]{\dfrac{\partial #1}{\partial #2}}
\newcommand{\ddfrq}[2]{\dfrac{\partial^2 #1}{\partial #2^2}}
\newcommand{\sfrac}[2]{\left.#1\middle/#2\right.}
\newcommand{\dsfr}[2]{\left.\partial #1\middle/\partial#2\right.}
\newcommand{\FL}[2]{#1\cdot10^{#2}}
\newcommand{\eps}{\varepsilon}
\newcommand{\hence}{\quad\Rightarrow\quad}
\newcommand{\imi}{\mathbf{i}}
\newcommand{\upx}[1]{{\mathrm{upx}\left[#1\right]}}
\newcommand{\upy}[1]{{\mathrm{upy}\left[#1\right]}}
\definecolor{ename-color}{rgb}{0.9, 0.9, 0.9}
\newcommand\lword[1]{\leavevmode\nobreak\hskip0pt plus\linewidth\penalty50\hskip0pt plus-\linewidth\nobreak#1}
\newcommand{\ename}[1]{\lword\colorbox{ename-color}{\Verb!#1!}}
\newcommand{\cvar}[1]{\lword\mintinline{text}{#1}}
\newcommand{\figref}[1]{рис.~\ref{#1}}
\newcommand{\secref}[1]{п.~\ref{#1}}
\newenvironment{shelloutput}%
  {\VerbatimEnvironment
    \begin{mdframed}[backgroundcolor=beige]
    \begin{Verbatim}}
  {\end{Verbatim}%
    \end{mdframed}}
\newcommand{\dt}{\triangle t}
\newcommand{\pluseq}{{{+}{=}}}
\newcommand{\minuseq}{{{-}{=}}}
\newcommand{\multeq}{{{*}{=}}}
\newcommand{\sminus}{\text{-}}
\newcommand{\arint}[3]{\displaystyle\int\limits_{#2}#1 \, d #3}
\newcommand{\feint}[1]{\arint{#1}{\Omega}{\vec x}}
\newcommand{\febint}[1]{\arint{#1}{\partial\Omega}{s}}
\newcommand\vecangle[2]{
  \setbox0=\hbox{$\!\vec{#1},\vec{#2}\!$}
  \ht0=\dimexpr\ht0-1pt\relax
  \widehat{\copy0}\,
}
\newcommand{\gvec}[1]{\left\{#1\right\}}
\newcommand{\quo}[1]{``#1''}
\newcommand{\const}{{\rm const}}
\newcommand{\mat}[1]{{\rm #1}}

\tcbset{
  osstyle/.style={
	colback=white,
	colframe=lightgray,
	coltitle=black,
	boxsep=2pt,
	arc=5pt,
	width=0.95\textwidth,
	title=Ubuntu,
	fonttitle=\bfseries,
	toptitle=3pt,
	bottomtitle=3pt
  },
}

\include{clisting}
\include{sub3}

\usepackage{tikz}
\usetikzlibrary{calc, arrows.meta, bending, decorations.pathreplacing}
\usepackage{witharrows}

\begin{document}

\newpage
\tableofcontents
\newpage

\begin{align}
&\dfr{u_i}{t} + U\frac{u_{i+1} - u_{i-1}}{2h} = 0,  \\
&\dfr{u_i}{t} + U\frac{u_{i} - u_{i-1}}{h} = 0
\end{align}
Тогда
\begin{align}
&\dfr{u_i}{t} = \frac{U}{2h}(u_{i-1} - u_i) - \frac{U}{2h}(u_{i+1} - u_i), \\
&\dfr{u_i}{t} = \frac{U}{h}(u_{i-1} - u_i)
\end{align}
Элементы матриц $\mat K$, $\mat L$, $\mat D$:
\begin{equation}
\begin{array}{rcccl}
& i-1 & i & i+1 & \\
\hline
(\mat K)_i = [ & \frac{U}{2h}  & 0 & -\frac{U}{2h} & ] \\
(\mat L)_i = [ & \frac{U}{h}  & -\frac{U}{h} & 0 & ] \\
(\mat D)_i = [ & \frac{U}{2h}  & -\frac{U}{h} & \frac{1}{2h} & ] \\
\end{array}
\end{equation}
Идея ограничения потока:
\begin{equation}
\mat M_L\dfr{u}{t} = \left(\mat K + \mat D - \mat F(u)\right)u
\end{equation}
В случае условия несжимаемости для скорости переноса можно записать
\begin{equation}
\label{eq:fem_tvd_init}
m_i\dfr{u_i}{t} = \sum_{j\neq i} \left(k_{ij} + d_{ij} - f_{ij}\right) (u_j - u_i) =
\sum_{j\neq i} \sigma_{ij} (u_j - u_i).
\end{equation}
Идея TVD-схем -- введём критерий экстремума $r_{ij}$ такой что
\begin{equation}
f_{ij} = \Phi(r_{ij}) d_{ij} > 0.
\end{equation}

$F(u)$ -- симметричный оператор $\hence f_{ij} = f_{ji}$.
Рассмотрим направленную связь $\overrightarrow{ij}$, такую что $k_{ij} < 0$.
Тогда:
\begin{align}
k_{ij} < 0, \\
k_{ji} > 0, \\
l_{ij} = 0, \\
l_{ji} > 0.
\end{align}
Для выполнения критерия Хартена уравнения
\cref{eq:fem_tvd_init}
в $j$-ой строке достаточно положить
\begin{equation}
f_{ij} < k_{ji} + d_{ji} = l_{ji}.
\end{equation}
Теперь рассмотрим строку $i$.
Очевидно, что формальный критерий Хартена $\sigma_{ij} \geq 0$ не выполняется:
\begin{equation}
\sigma_{ij} = -f_{ij} = -\Phi(r_{ij}) d_{ij} < 0
\end{equation}
Выберем $\Phi(r_{ij})$ таким образом, чтобы антидиффузный поток из узла $j$ в узел $i$
можно было разложить в сумму диффузных:
\begin{equation}
\sigma_{ij} (u_j - u_i) = -f_{ij} (u_j - u_i) = \sum_{k\neq i} f^*_{ik}(u_k - u_i)
\end{equation}

Во первых зададим симметричность функции ограничителя:
\begin{equation}
\Phi(r) = r \Phi(1/r)
\end{equation}
Тогда
\begin{equation}
-f_{ij}(u_j - u_i) = -\Phi(r_{ij}) (u_j - u_i) = -\Phi(1/r_{ij}) r_{ij} d_{ij} (u_j - u_i)
\end{equation}
Задача выразить $r_{ij}$ через $r^*_{ik} \geq 0$
\begin{equation}
\Delta u_{ij} = -r_{ij}(u_j - u_i) = \sum_k r^*_{ijk}(u_k - u_i)
\end{equation}

\subsection{Интерполяционные методы}

Из свойств ограничителя $r_{ij} \geq 0$.
В одномерном случае $j=i+1$ -- узел по потоку. 
\begin{equation}
r_{ij} = r_{i,i+1} = \frac{u_i - u_{i-1}}{u_{i+1} - u_{i}}
\hence
\Delta u_{i,i+1} = u_{i-1} - u_{i}.
\end{equation}
В многомерном случае
\begin{equation}
r_{ij} = \frac{u_i - u^*}{u_j - u_{i}}
\end{equation}

TODO

\subsection{Алгебраический метод}
Обобщим одномерный критерий TVD:
\begin{equation}
r_{i,i+1} = \frac{k_{i,i-1} (u_{i-1} - u_{i})}{k_{i,i+1} (u_{i+1} - u_{i})}
\end{equation}

Наивное многомерное обобщение могло бы выглядеть так
\begin{equation}
r_{ij} = \frac
{\displaystyle\sum_k k^+_{ik} (u_{k} - u_{i})}
{\displaystyle\sum_k k^-_{ik} (u_{k} - u_{i})}
\end{equation}

Тогда запишем разность
\begin{equation}
\Delta u_{ij} = -r_{ij}(u_j - u_i) = \sum_k r^*_{ijk}(u_k - u_i),
\end{equation}
где
\begin{equation}
r^*_{ijk} = \frac{-k_{ik}^+ (u_j - u_i)} {\displaystyle\sum_k k_{ik}^- (u_k - u_i)}
\end{equation}

Такое определение не гарантирует положительность $r_{ijk}$.
Необходимо, чтобы знаки $(u_j - u_i)$ и $(u_k - u_i)$ совпадали.
Заметим ещё одно свойство одномерной схемы:
в случае монотонного поведения $u_i$, знаки разностей $(u_{i+1} - u_i)$, $(u_{i-1} - u_i)$ в числителе и знаменателе различаются.
Причём знак в знаменателе совпадает со знаком текущего узла по потоку $j$ (который в одномерном случае равен $i+1$).
Тогда
\begin{equation}
r_{ij} = \begin{cases}
\frac {\displaystyle\sum_k k^+_{ik} (u_{k} - u_{i})^-} {\displaystyle\sum_k k^-_{ik} (u_{k} - u_{i})^-}, &\quad u_i \geq u_j, \\[30pt]
\frac {\displaystyle\sum_k k^+_{ik} (u_{k} - u_{i})^+} {\displaystyle\sum_k k^-_{ik} (u_{k} - u_{i})^+}, &\quad u_i < u_j.
\end{cases}
\end{equation}

Окончательно запишем
\begin{equation}
r_{ij} = \begin{cases}
Q_i^+ / P_i^+, &\quad u_i \geq u_j,\\
Q_i^- / P_i^-, &\quad u_i \geq u_i,\\
\end{cases}
\end{equation}

\begin{align}
&Q_i^{\pm} = \sum_k k_{ik}^+(u_k - u_i)^{\pm}, \\
&P_i^{\pm} = \sum_k k_{ik}^-(u_k - u_i)^{\mp}.
\end{align}

Вспомитная строку $j$ окончательно запишем антидиффузию:
\begin{equation}
f_{ij} = f_{ji} = \min(\Phi(r_{ij}) d_{ij}, l_{ji})
\end{equation}


Алгоритм будет иметь такой вид
\begin{enumerate}
\item
Вычислить $k_{ij}$ -- сеточную матрицу переноса второго порядка,
\item
Вычислить линейную диффузию $d_{ij} = \max(0, -k_{ij}, -k_{ji})$.
\item
Вычислить $l_{ij} = k_{ij} + d_{ij}$
\item
Для каждого узла $i$ определить $Q^{\pm}_i$, $P^{\mp}_i$
\item
В цикле для напраленных граней $\overrightarrow{ij}$ определить $f_{ij}$
\item
окончательно собрать матрицу переноса с ограничением
$\sigma_{ij} = k_{ij} + d_{ij} - f_{ij}$.

\end{enumerate}

\end{document}
