\subsection{Работа с системой контроля версий}
Работать с гитом можно как с хоста (из папки \ename{CFDCourse26}),  так и из контейнера (из папки \ename{\app}).
Ниже будут даны инструкции для работы с гитом в консоли.
Альтернативно, можно установить графический интерфейс (например \ename{GitExtensions} для Windows)
или командами vscode на вкладке \ename{Source Control}.

Из системы контроля версий исключены следующие каталоги:
\begin{itemize}
\item build*/   -- папки со сборками,
\item .vscode, .vscode-server -- настройки и рисширения vscode,
\item local\_data -- папка для хранения любых пользовательских данных.
\end{itemize}
Изменения из этих папках не будут отслежены и скоммичены.

\subsubsection{Порядок работы с репозиторием CFDCourse}

Основная ветка проекта -- \ename{master}. После каждой лекции в эту ветку будет отправлен коммит с сообщением \ename{lect{index}}.
В этом коммите будет дополнен pdf документ с содержанием лекции, задание по итогам лекции и необходимые для этого задания изменения в коде.

\subsubsubsection{Получение последнего коммита}
Таким образом, {\bf после лекции}, после того, как изменение \ename{lect{index}} придёт на сервер, необходимо выполнить следующие команды
\begin{shelloutput}
git checkout master  # перейти на основную ветку
git pull             # получить изменения
\end{shelloutput}

Если изменения не содержали никаких изменений в настроечных файлах контейнера \ename{Dockerfile, docker-compose.yaml},
то для сборки проекта рестарт контейнера не требуется.
Иначе требуется пересобрать контейнер:
\begin{shelloutput}
docker stop cfd26            # остановить текущий контейнер
docker compose up --build -d # пересобрать новый
\end{shelloutput}

\subsubsubsection{Создание коммита с текущим дз}
{\bf Перед началом лекции}, если была сделана какая то работа по заданиям,
\begin{shelloutput}
> git checkout -b hw-lect{index}      # создать локальную ветку, содержащую задание
> git add .
> git commit -m "{свой комментарий}"  # скоммитить свои изменения в эту ветку
\end{shelloutput}

Даже если задание выполнено не до конца, вы в любой момент можете переключиться на ветку с заданием и его доделать
\begin{shelloutput}
git checkout work-lect{index}
\end{shelloutput}

\subsubsubsection{Создание коммита с прошлым дз}
Если вы не сделали задание вовремя и решили вернутся к нему позже, то нужно
\begin{shelloutput}
git checkout master            # перейти на основную ветку
git log --oneline              # в списке всех коммитов найти хэш коммита
                               # lect{index} той лекции которую нужно сделать
git checkout <...>             # переключиться на этот коммит по его хэшу
git checkout -b hw-lect{index} # создать ветку от этого коммита и работать в этой ветке
...
git commit -m "comment"        # по окончании работы скоммитить изменения
git checkout master            # и вернуться в основную ветку
\end{shelloutput}

