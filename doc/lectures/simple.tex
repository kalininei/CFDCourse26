\section{Моделирирование течения вязкой несжимаемой жикости}
\subsection{Система уравнений Навье-Стокса}

Будем рассматривать стационарную двумерную систему уравнений
Навье-Стокса для вязкой несжимаемой жидкости.
В безразмерном консервативном виде в декартовой системе координат она имеет вид

\begin{align}
    \label{eq:ns2d_u}
    & \dfr{u^2}{x} + \dfr{uv}{y} =
        -\dfr{p}{x}
        + \frac{1}{\Ren}\left(\dfrq{u}{x} + \dfrq{u}{y}\right),\\[5pt]
    \label{eq:ns2d_v}
    & \dfr{uv}{x} + \dfr{v^2}{y} =
        -\dfr{p}{y}
        + \frac{1}{\Ren}\left(\dfrq{v}{x} + \dfrq{v}{y}\right), \\[5pt]
    \label{eq:ns2d_div}
    &\dfr{u}{x} + \dfr{v}{y} = 0.
\end{align}

Неизвестными являются поля скорости: $u$ -- в направлении оси $x$,
$v$ -- в направлении оси $y$, и давления $p$.

Число Рейнольдса определено через характерную скорость $U$, [м/c] и
характерный линейный размер $L$, [м] как
\begin{equation*}
    \Ren = \frac{UL\rho}{\mu},
\end{equation*}
где $\rho$, [кг/м$^3$] -- постоянная (вследствии несжимаемости) плотность жидкости, а
$\mu$, [Па$\cdot$с] -- динамическая вязкость жидоксти.

Характерое значение для давление выпишется в виде:
$ p^0 = \rho U^2 $, [Па].

Для решения этой системы будем использовать метод конечных разностей
с аппроксимацией по пространству второго порядка и последовательное (раздельное) решение входящих в неё
уравнений.

Глядя на вид уранений \cref{eq:ns2d_u,eq:ns2d_v,eq:ns2d_div}
можно выделить несколько проблем, которые необходимо решить
при построении расчётной схемы:
\begin{itemize}
\item нелинейность конвективного оператора в \eqref{eq:ns2d_u}, \eqref{eq:ns2d_v},
\item отсутствие явного уравнения для определения давления,
\item аппркосимация первых производных для давления и скорости со вторым порядком точности.
\end{itemize}


Для решения первой проблемы будем использовать итерационный процесс с линеаризацией -- то есть
записывать уравнение на итерационном слое используя значения неизвестных полей с прошлого слоя.
Вторую проблему будем решать с помощью алгоритма SIMPLE
связывания давления и скорости (Pressure-Velocity Coupling).

\subsection{Схема расчёта SIMPLE}
\label{sec:simple_scheme}

\subsubsection{Линеаризация}
Для избавления от нелинейности по скорости в системе \cref{eq:ns2d_u,eq:ns2d_div} в качестве скорости переноса будем
использовать значение с прошлой итерации.
Тогда задача на одном итерационном слое примет вид

\begin{align}
    \label{eq:ns2d_semi_u}
    & \dfr{u \hat u}{x} + \dfr{v \hat u}{y} =
        -\dfr{\hat p}{x}
        + \frac{1}{\Ren}\left(\dfrq{\hat u}{x} + \dfrq{\hat u}{y}\right),\\[5pt]
    \label{eq:ns2d_semi_v}
    & \dfr{u\hat v}{x} + \dfr{v \hat v}{y} =
        -\dfr{\hat p}{y}
        + \frac{1}{\Ren}\left(\dfrq{\hat v}{x} + \dfrq{\hat v}{y}\right), \\[5pt]
    \label{eq:ns2d_semi_div}
    &\dfr{\hat u}{x} + \dfr{\hat v}{y} = 0.
\end{align}

На итерационном слое значения $u, v, p$
известны, а $\hat u, \hat v, \hat p$ подлежат определению.

Критерием выхода из итерационного процесса является пороговое условие на невязку,
вычисленную с использованием найденных на слое значений неизвестных:
\begin{align}
    \nonumber
    &r_u = \dfr{\hat u \hat u}{x} + \dfr{\hat u \hat v}{y}
        + \dfr{\hat p}{x}
        - \frac{1}{\Ren}\left(\dfrq{\hat u}{x} + \dfrq{\hat u}{y}\right),\\[5pt]
    \nonumber
    &r_v = \dfr{\hat u\hat v}{x} + \dfr{\hat v \hat v}{y}
        +\dfr{\hat p}{y}
        - \frac{1}{\Ren}\left(\dfrq{\hat v}{x} + \dfrq{\hat v}{y}\right), \\[5pt]
    \label{eq:ns2d_residual}
    &\max(\lVert r_u \rVert, \lVert r_v \rVert) < \eps.
\end{align}

\subsubsection{Релаксация по скорости}
\label{sec:simple-algo}

В результате пространственной аппроксимации уравнений \cref{eq:ns2d_u}
получим матричную систему вида
\begin{align}
\nonumber
&\mat{A}^u \hat u = -\dfr{\hat p}{x}. \\[5pt]
\nonumber
&\mat{A}^v \hat v = -\dfr{\hat p}{y}.
\end{align}
Отметим, что хотя матрицы $\mat{A}^u$ и $\mat{A}^v$ получены в результате
аппроксимации одного и того же дифференциального оператора (конвекции и диффузии),
они могут различаться из-за особенностей аппроксимации.

Эта система при малых числах $\Ren$  не имеет диагонального преобладания
и неудобна для численного решения. Чтобы исправить этот недостаток
введём релаксацию по диагональному параметру с коэффициентом $\alpha_u$:
\begin{equation}
\label{eq:ns2d_linearized_u}
\frac{1}{\alpha_u} a^u_{ii}\hat u_i = -\sum_{j\neq i} a^u_{ij} \hat u_j -\dfr{\hat p}{x} + \frac{\alpha_u - 1}{\alpha_u} a^u_{ii} u_i
\end{equation}

\subsubsection{Связывание давления и скорости}
Приведём алгоритм для явного выражения уравнения для давления из
уравнения неразрывности \eqref{eq:ns2d_semi_div}.

Распишем искомые перенные в виде суммы 
\begin{equation}
    \label{eq:ns2d_decomp}
\begin{array}{l}
    \hat u = u^* + u',\\
    \hat v = v^* + v',\\
    \hat p = p + p'.
\end{array}
\end{equation}

Пусть введённое выше поле $u^*$ удовлетворяет уравнению
\begin{align}
    \label{eq:ns2d_ustar}
    &\frac{1}{\alpha_u} a^u_{ii} u^*_i = -\sum_{j\neq i} a^u_{ij} u^*_j - \dfr{p}{x} + \frac{\alpha_u - 1}{\alpha_u} a^u_{ii} u_i, \\[5pt]
    \label{eq:ns2d_vstar}
    &\frac{1}{\alpha_u} a^v_{ii} v^*_i = -\sum_{j\neq i} a^v_{ij} v^*_j - \dfr{p}{y} + \frac{\alpha_u - 1}{\alpha_u} a^v_{ii} v_i,
\end{align}

Тогда уравнение для поправки скорости запишем вычтя последнее выражение из
уравнения \eqref{eq:ns2d_linearized_u}:
\begin{align}
  \label{eq:ns2d_uprime}
  &\frac{1}{\alpha_u} a^u_{ii} u'_i = -\sum_{j\neq i} a^u_{ij} u'_j - \dfr{p'}{x}, \\[5pt]
  \label{eq:ns2d_vprime}
  &\frac{1}{\alpha_u} a^v_{ii} v'_i = -\sum_{j\neq i} a^v_{ij} v'_j - \dfr{p'}{y},
\end{align}

Основная идея алгоритма SIMPLE заключается в приближённом представлении выражения \eqref{eq:ns2d_uprime}
в явном виде относительно поправки. Для этого отбросим внедиагональные компоненты в матрице $\mat A$.
Тогда можно явно выразить поправку скорости как
\begin{equation}
    \label{eq:ns2d_uprime_approx}
    u'_i \approx -d^u_i \left(\dfr{p'}{x}\right)_i, \quad d^u_i = \frac{\alpha_u}{a^u_{ii}}
\end{equation}

Аналогичные рассуждения в отношении поправки поперечной скорости $v'$ приводят к выражению

\begin{equation}
    \label{eq:ns2d_vprime_approx}
    v'_i \approx -d^v_i \left(\dfr{p'}{y}\right)_i, \quad d^v_i = \frac{\alpha_u}{a^v_{ii}}
\end{equation}

Далее используем уравнение неразрывности \cref{eq:ns2d_semi_div}. Подставим
в него разложения \cref{eq:ns2d_decomp} и используем
\cref{eq:ns2d_uprime_approx,eq:ns2d_vprime_approx}.
Тогда
получим уравнение Пуассона с непостоянным по пространству векторным коэффициентом диффузии $\left(d^u, d^v\right)$
относительно поправки давления $p'$:
\begin{equation}
    \label{eq:ns2d_pprime_diff}
    -\left[
    \dfr{}{x}\left(
       d^u \dfr{p'}{x} 
            \right)
    +\dfr{}{y}\left(
       d^v \dfr{p'}{y} 
            \right)
    \right]
    =
    -\left(\dfr{u^*}{x} + \dfr{v^*}{y}\right).
\end{equation}

\subsubsection{Итерационный процесс}
Определим порядок вычислений на итерационном слое.
Напомним, что значения $u, v, p$ с
предыдущего слоя нам известно и задача
состоит в нахождении значений $\hat u, \hat v, \hat p$
на текущем слое.

\begin{enumerate}
\item Из уравнений \eqref{eq:ns2d_ustar}, \eqref{eq:ns2d_vstar}
      вычисляются значения $u^*, v^*$;
\item Они используются для вычисления правой части уравнения \eqref{eq:ns2d_pprime_diff},
      в результате решения которого находится поправка давления $p'$;
\item Дифференцируя найденную поправку давления найдём поправки скорости $u', v'$
      из выражений \eqref{eq:ns2d_uprime_approx}, \eqref{eq:ns2d_vprime_approx};
\item Окончательно выразим значения переменных для текущего слоя из \eqref{eq:ns2d_decomp}.
      Для улучшения стабильности алгоритма значение давления вычисляют
      с некоторым коэффициентом релаксации $\alpha_p$:
      \begin{equation*}
           \hat p = p + \alpha_p p';
      \end{equation*}
\item Далее проводится вычисление невязки с ипользованием найденных значений $\hat u, \hat v, \hat p$
      из выражения \eqref{eq:ns2d_residual}. Если она недостаточно мала,
      то выполняется присваивание
      $
           u = \hat u, \; v=\hat v, \; p = \hat p
      $ 
      и возвращение на шаг 1.
\end{enumerate}

Полученные на каждом шаге итерационного процесса компоненты скорости $\hat u, \hat v$
точно удовлетворяют уравнению неразрывности \eqref{eq:ns2d_semi_div}, но
уравнения движения \eqref{eq:ns2d_semi_u}, \eqref{eq:ns2d_semi_v} выполняются лишь приближённо.

Всего в алгоритме SIMPLE есть два параметра: коэффициент
релаксации давления $\alpha_p$ и коэффициент релаксации скорости $\alpha_u$.
Характерые значения для этих параметров:
\begin{equation*}
\alpha_u = 0.8, \quad \alpha_p = 0.3.
\end{equation*}

\subsection{Пространственная аппроксимация методом конечных разностей}
\label{sec:simple_scheme_fdm}

Для численной реализации алгоритма решения
необходимо провести пространственную аппроксимацию полудискретизованных
выражений
\eqref{eq:ns2d_ustar}, \eqref{eq:ns2d_vstar}, \eqref{eq:ns2d_uprime_approx},
\eqref{eq:ns2d_vprime_approx}, \eqref{eq:ns2d_pprime_diff}.

Для решения проблем с неустойчивостью схемы по давлению (checkboard instability)
проводить конечноразностную аппроксимацию будем на разнесённой сетке (Staggered Grid).

\subsubsection{Разнесённая сетка}

Будем использовать структурированную четырёхугольную сетку
с постоянным шагом по пространству.
При этом неизвестные параметры будем задавать
по схеме, представленной на \figref{fig:staggered_grid}.

\begin{figure}[h]
\centering
\includegraphics[width=0.6\linewidth]{staggered_grid.png}
\caption{Разнесённая сетка}
\label{fig:staggered_grid}
\end{figure}

Введём разбиение сетки:
$n_x$ -- количество ячеек в направлении $x$,
$n_y$ -- количество ячеек в направлении $y$.

Очевидно, что при использовании такого разнесённого шаблона,
количество точек, в которых заданы значения, будет
различным для разных параметров.
Так количество узловых значений давления будет равно $n_x \times n_y$,
продольной скорости $u$ -- $(n_x+1) \times n_y$, а поперечной $v$ -- $n_x \times (n_y+1)$.

Использование такого расположения узловых точек
даёт преимущество при аппроксимации
первых производных. Так, конечная разность
\begin{equation*}
\left.\dfr{p}{x}\right|_{i, j+\tfrac12} = \frac{p_{i+\tfrac12,j+\tfrac12} - p_{i-\tfrac12,j+\tfrac12}}{h_x} + o(h_x^2)
\end{equation*}
будем симметричной в узле $i,j+\tfrac12$, где задана
компонента скорости $u$, и поэтому будет иметь там второй порядок точности.

Выражения \eqref{eq:ns2d_ustar}, \eqref{eq:ns2d_uprime_approx} аппроксимируются
на сетке для $u$, выражения \eqref{eq:ns2d_vstar}, \eqref{eq:ns2d_vprime_approx} -- 
на сетке для $v$, а \eqref{eq:ns2d_pprime_diff} -- на сетке для $p$.

Введём сквозную линейную нумерацию узлов сетки: нулевой узел разположим в левом нижнем углу,
далее будем индексировать слева направа и потом снизу вверх.
Для основной сетки перевод двумерного индекса $i,j$ в сквозной индекс будет проводится по формуле
\begin{equation}
    \label{eq:ns2d_kij}
    k(i,j) = j(n_x+1)+i.
\end{equation}
Для сеток, на которых заданы сеточные параметры, такой перевод примет вид
\begin{align}
    \label{eq:ns2d_kipjp}
    &k(i+\tfrac12,j+\tfrac12) = jn_x + i, \quad - \; \text{сетка для давления } p\\[10pt]
    \label{eq:ns2d_kijp}
    &k(i,j+\tfrac12) = j(n_x+1) + i, \quad - \;  \text{сетка для продольной скорости } {\color{red} u} \\[10pt]
    \label{eq:ns2d_kipj}
    &k(i+\tfrac12,j) = jn_x + i, \quad - \; \text{сетка для поперечной скорости } {\color{blue} v}
\end{align}

\subsubsection{Уравнения движения}

Запишем конечноразностную аппроксимацию уравнения \eqref{eq:ns2d_ustar} для пробной скорости $u^*$ в ``красных'' узлах сетки
$\left(i, j+\tfrac12\right)$:

\begin{align}
    \label{eq:ns2d_ustar_discr}
       &\frac{1}{\alpha_u}\frac{2}{\Ren}\left(\frac{1}{h_x^2} + \frac{1}{h_y^2}\right)
          \left(u^*_{i,j+\tfrac12}\right)   \\[10pt]
    \nonumber
       &-\frac{1}{\Ren}\frac{1}{h_x^2}
          \left(
            u^*_{i-1,j+\tfrac12} + u^*_{i+1,j+\tfrac12}
          \right)
       -\frac{1}{\Ren}\frac{1}{h_y^2}
          \left(
            u^*_{i,j-\tfrac12} + u^*_{i,j+\tfrac32}
          \right)   \\[10pt]
    \nonumber
       &+\frac{1}{h_x}
          \left(
            \left(u u^*\right)_{i+\tfrac12,j+\tfrac12}-
            \left(u u^*\right)_{i-\tfrac12,j+\tfrac12}
          \right)
        +\frac{1}{h_y}
          \left(
            \left(v u^*\right)_{i,j+1}-
            \left(v u^*\right)_{i,j}
          \right)   \\[10pt]
    \nonumber
       =&\frac{1-\alpha_u}{\alpha_u}\frac{2}{\Ren}\left(\frac{1}{h_x^2} + \frac{1}{h_y^2}\right)
          \left(u_{i,j+\tfrac12}\right)   \\[10pt]
    \nonumber
       &-\frac{1}{h_x}\left(p_{i+\tfrac12,j+\tfrac12} - p_{i-\tfrac12,j+\tfrac12}\right).
\end{align}

В приведённом выражении
за исключением конвективных слагаемых вида $u u$ все остальные
сеточные вектора используются на своих сетках.
Конвективные слагаемые распишем через полусуммы вида:
\begin{equation*}
    u_{i+\tfrac12} = \frac{u_{i} + u_{i+1}}{2} + o(h^2)
\end{equation*}

Тогда
\begin{align*}
    \left(u u^*\right)_{i+\tfrac12,j+\tfrac12} =&
        \left(u_{i+\tfrac12,j+\tfrac12} \vphantom{u^*_{\tfrac12}}\right)\left(u^*_{i+\tfrac12,j+\tfrac12}\right) =
        \frac14 \left(u_{i, j+\tfrac12} + u_{i+1,j+\tfrac12}\vphantom{u^*_{\tfrac12}}\right)
                \left(u^*_{i, j+\tfrac12} + u^*_{i+1,j+\tfrac12}\right), \\[10pt]
    \left(u u^*\right)_{i-\tfrac12,j+\tfrac12} =&
        \frac14 \left(u_{i, j+\tfrac12} + u_{i-1,j+\tfrac12}\vphantom{u^*_{\tfrac12}}\right)
                \left(u^*_{i, j+\tfrac12} + u^*_{i-1,j+\tfrac12}\right), \\[10pt]
    \left(v u^*\right)_{i,j+1} =&
        \frac14 \left(v_{i+\tfrac12, j+1} + v_{i-\tfrac12,j+1}\vphantom{u^*_{\tfrac12}}\right)
                \left(u^*_{i, j+\tfrac32} + u^*_{i,j+\tfrac12}\right), \\[10pt]
    \left(v u^*\right)_{i,j} =&
        \frac14 \left(v_{i+\tfrac12, j} + v_{i-\tfrac12,j}\vphantom{u^*_{\tfrac12}}\right)
                \left(u^*_{i, j+\tfrac12} + u^*_{i,j-\tfrac12}\right).
\end{align*}

Схему \eqref{eq:ns2d_vstar}
можно записать в виде системы линейных уравнений вида
\begin{equation}
    \label{eq:ns2d_ustar_slae}
    \mat{A}^u u^* = b^{u},
\end{equation}
Сеточная матрица $A^u$ будет иметь $(n_x+1)n_y$ строк.
Для строки, соответствующей $\left(i,j+\tfrac12\right)$ узлу ненулевыми будут столбцы,
соответствующие узлам:
\begin{itemize}
\item $\left(i,j+\tfrac12\right)$,
\item $\left(i+1,j+\tfrac12\right)$,
\item $\left(i-1,j+\tfrac12\right)$,
\item $\left(i,j+\tfrac32\right)$,
\item $\left(i,j-\tfrac12\right)$.
\end{itemize}

В случае использования стандартной нумерации узлов структурированной сетки,
когда нулевой индекс соответствуют левому нижнему узлу и далее нумерация идёт
с быстрым индексом $i$, то матрица будет пятидиагональной.


Подставим полученные выражения в конвективную часть выражения \eqref{eq:ns2d_ustar_discr}.
Множитель при диагональном элементе $u^*_{i,j+\tfrac12}$ будет равен:
\begin{align*}
        \frac{\tau}{4}\Biggl(
         \underbrace{\frac{u_{i,j+\tfrac12} - u_{i-1,j+\tfrac12}}{h_x}}_{\left.\ddfr{u}{x}\right|_{i-\tfrac12,j+\tfrac12}}
        +\underbrace{\frac{u_{i+1,j+\tfrac12} - u_{i,j+\tfrac12}}{h_x}}_{\left.\ddfr{u}{x}\right|_{i+\tfrac12,j+\tfrac12}}
        +\underbrace{\frac{v_{i+\tfrac12,j+1} - v_{i+\tfrac12,j}}{h_y}}_{\left.\ddfr{v}{y}\right|_{i+\tfrac12,j+\tfrac12}}
        +\underbrace{\frac{v_{i-\tfrac12,j+1} - v_{i-\tfrac12,j}}{h_y}}_{\left.\ddfr{v}{y}\right|_{i-\tfrac12,j+\tfrac12}}
        \Biggr)
\end{align*}

Сумма первого и четвёртого слагаемых представляет собой разностный
аналог уравнения неразрывности \eqref{eq:ns2d_semi_div}, записанной для ``чёрного'' узла
сетки $i-\tfrac12, j+\tfrac12$ относительно компонент
скорости с предыдущей итерации.
Как было сказано ранее,
в настоящем алгоритме
уравнение неразрывности для итоговых по результатам итерации скорости в этих узлах выполняется точно.
Поэтому эта сумма в точности будет равна нулю. Аналогичный результат получится
и для суммы второго и третьего слагаемых. Отсюда следует вывод, что
конвективное слагаемое не даёт вклад в диагональ итоговой матрицы (как и следовало ожидать от симметричной аппроксимации).

Окончательно запишем все пять ненулевых вхождений в строку матрицы:
\begin{align}
    \label{eq:ns2d_au}
    \mat{A}^u\left[
        k\left(i,j+\tfrac12\right),
        k\left(i,j+\tfrac12\right)\right]
        &= \frac{1}{\alpha_u}\frac{2}{\Ren}\left(\frac{1}{h_x^2} + \frac{1}{h_y^2}\right)
        \quad - \;\text{основная диагональ}, \\[10pt]
    \nonumber
    \mat{A}^u\left[
        k\left(i,j+\tfrac12\right),
        k\left(i+1,j+\tfrac12\right)\right]
        &= -\frac{1}{\Ren}\frac{1}{h_x^2}
           +\frac{1}{4h_x}\left(u_{i,j+\tfrac12}+u_{i+1,j+\tfrac12}\right)
        \quad - \;\text{первая верхняя диагональ}, \\[10pt]
    \nonumber
    \mat{A}^u\left[
        k\left(i,j+\tfrac12\right),
        k\left(i-1,j+\tfrac12\right)\right]
        &= -\frac{1}{\Ren}\frac{1}{h_x^2}
           -\frac{1}{4h_x}\left(u_{i,j+\tfrac12}+u_{i-1,j+\tfrac12}\right)
        \quad - \;\text{первая нижняя диагональ}, \\[10pt]
    \nonumber
    \mat{A}^u\left[
        k\left(i,j+\tfrac12\right),
        k\left(i,j+\tfrac32\right)\right]
        &= -\frac{1}{\Ren}\frac{1}{h_y^2}
           +\frac{1}{4h_y}\left(v_{i+\tfrac12,j+1}+v_{i-\tfrac12,j+1}\right)
        \quad - \;\text{вторая верхняя диагональ}, \\[10pt]
    \nonumber
    \mat{A}^u\left[
        k\left(i,j+\tfrac12\right),
        k\left(i,j-\tfrac12\right)\right]
        &= -\frac{1}{\Ren}\frac{1}{h_y^2}
           -\frac{1}{4h_y}\left(v_{i+\tfrac12,j}+v_{i-\tfrac12,j}\right)
        \quad - \;\text{вторая нижняя диагональ}.
\end{align}
Правая часть аппроксимируется в виде
\begin{equation*}
    b^{u}[k(i, j+\tfrac12)] = - \frac{1}{h_x}\left(p_{i+\tfrac12,j+\tfrac12} - p_{i-\tfrac12,j+\tfrac12}\right)
        + \frac{\alpha_u-1}{\alpha_u}\frac{2}{\Ren}\left(\frac{1}{h_x^2} + \frac{1}{h_y^2}\right) u_{i, j+\tfrac12}.
\end{equation*}
Здесь $k(i,j)$ -- функция перевода двумерного индекса в сквозной \eqref{eq:ns2d_kijp}.

Аналогичные выкладки для второго из уравнений движения \eqref{eq:ns2d_vstar}
дают систему уравнений
\begin{equation}
    \label{eq:ns2d_vstar_slae}
    \mat{A}^v v^* = b^{v},
\end{equation}
элементы пятидиагональной матрицы которой имеют вид
\begin{align}
    \label{eq:ns2d_av}
    \mat{A}^v\left[
        k\left(i+\tfrac12,j\right),
        k\left(i+\tfrac12,j\right)\right]
        &= \frac{1}{\alpha_u}\frac{2}{\Ren}\left(\frac{1}{h_x^2} + \frac{1}{h_y^2}\right)
        \quad - \;\text{основная диагональ}, \\[10pt]
    \nonumber
    \mat{A}^v\left[
        k\left(i+\tfrac12,j\right),
        k\left(i+\tfrac32,j\right)\right]
        &= -\frac{1}{\Ren}\frac{1}{h_x^2}
           +\frac{1}{4h_x}\left(u_{i+1,j+\tfrac12}+u_{i+1,j-\tfrac12}\right)
        \quad - \;\text{первая верхняя диагональ}, \\[10pt]
    \nonumber
    \mat{A}^v\left[
        k\left(i+\tfrac12,j\right),
        k\left(i-\tfrac12,j\right)\right]
        &= -\frac{1}{\Ren}\frac{1}{h_x^2}
           -\frac{1}{4h_x}\left(u_{i,j+\tfrac12}+u_{i,j-\tfrac12}\right)
        \quad - \;\text{первая нижняя диагональ}, \\[10pt]
    \nonumber
    \mat{A}^v\left[
        k\left(i+\tfrac12,j\right),
        k\left(i+\tfrac12,j+1\right)\right]
        &= -\frac{1}{\Ren}\frac{1}{h_y^2}
           +\frac{1}{4h_y}\left(v_{i+\tfrac12,j}+v_{i+\tfrac12,j+1}\right)
        \quad - \;\text{вторая верхняя диагональ}, \\[10pt]
    \nonumber
    \mat{A}^v\left[
        k\left(i+\tfrac12,j\right),
        k\left(i+\tfrac12,j-1\right)\right]
        &= -\frac{1}{\Ren}\frac{1}{h_y^2}
           -\frac{1}{4h_y}\left(v_{i+\tfrac12,j}+v_{i+\tfrac12,j-1}\right)
        \quad - \;\text{вторая нижняя диагональ}.
\end{align}
Правая часть аппроксимируется в виде
\begin{equation*}
    b^{v}[k(i+\tfrac12, j)] =  - \frac{1}{h_y}\left(p_{i+\tfrac12, j+\tfrac12} - p_{i+\tfrac12,j-\tfrac12}\right)
        + \frac{\alpha_u-1}{\alpha_u}\frac{2}{\Ren}\left(\frac{1}{h_x^2} + \frac{1}{h_y^2}\right) v_{i, j+\tfrac12}.
\end{equation*}
Используется функция перевода двумерного индекса в сквозной из \eqref{eq:ns2d_kipj}.

\subsubsection{Уравнение для поправки давления}
Распишем уравнение \eqref{eq:ns2d_pprime_diff}
на ``чёрной'' сетке методом конечных разностей.
Для первого слагаемого получим
\begin{equation}
\label{eq:ns2d_d2pdx2}
\begin{array}{ll}
\left.\ddfr{}{x}\left(d^u \ddfr{p'}{x}\right) \right|_{i+\tfrac12, j+\tfrac12}
    &\approx
        \dfrac{1}{h_x}\left(
            d^u_{i+1, j+\tfrac12} \left. \ddfr{p'}{x} \right|_{i+1, j+\tfrac12} -
            d^u_{i, j+\tfrac12} \left. \ddfr{p'}{x} \right|_{i, j+\tfrac12}
        \right) \\[10pt]

    &=
        \dfrac{1}{h_x}\left(
            d^u_{i+1, j+\tfrac12} \dfrac{p'_{i+\tfrac32,j+\tfrac12} - p'_{i+\tfrac12,j+\tfrac12}}{h_x} - 
            d^u_{i, j+\tfrac12}  \dfrac{p'_{i+\tfrac12,j+\tfrac12} - p'_{i-\tfrac12,j+\tfrac12}}{h_x}
        \right).
\end{array}
\end{equation}
Коэффициенты $d^u$ и $d^v$ определяются для ``красной'' и ``синей'' сеток через диагональные значения матриц $\mat A$:
\begin{align}
\label{eq:ns2d_du_fdm}
d^u_{i,j+\tfrac12} =    \sfrac{1}{\mat{A}^u\left[
        k\left(i,j+\tfrac12\right),
        k\left(i,j+\tfrac12\right)\right]}, \\[5pt]
\label{eq:ns2d_dv_fdm}
d^v_{i+\tfrac12,j} =    \sfrac{1}{\mat{A}^v\left[
        k\left(i+\tfrac12,j\right),
        k\left(i+\tfrac12,j\right)\right]}.
\end{align}
Аналогично расписываются остальные слагаемые. В результате получим систему линейных уравнений вида
\begin{equation}
    \label{eq:ns2d_pprime_slae}
    A^p p' = b^p,
\end{equation}
где ненулевые коэффициенты пятидиагональной матрицы примут вид
\begin{align}
    \label{eq:ns2d_ap}
    A^p[k(i+\tfrac12, j+\tfrac12), k(i+\tfrac12, j+\tfrac12)] =&
        \frac{1}{h_x^2}\left( d^u_{i+1,j+\tfrac12} + d^u_{i,j+\tfrac12} \right)
        +\frac{1}{h_y^2}\left( d^v_{i+\tfrac12,j} + d^v_{i+\tfrac12,j+1} \right), \\[10pt]
    \nonumber
    A^p[k(i+\tfrac12, j+\tfrac12), k(i+\tfrac32, j+\tfrac12)] =&
        -\frac{1}{h_x^2}d^u_{i+1,j+\tfrac12}, \\[10pt]
    \nonumber
    A^p[k(i+\tfrac12, j+\tfrac12), k(i-\tfrac12, j+\tfrac12)] =&
        -\frac{1}{h_x^2}d^u_{i,j+\tfrac12}, \\[10pt]
    \nonumber
    A^p[k(i+\tfrac12, j+\tfrac12), k(i+\tfrac12, j+\tfrac32)] =&
        -\frac{1}{h_y^2}d^v_{i+\tfrac12,j+1}, \\[10pt]
    \nonumber
    A^p[k(i+\tfrac12, j+\tfrac12), k(i+\tfrac12, j-\tfrac12)] =&
        -\frac{1}{h_y^2}d^v_{i+\tfrac12,j}.\\[10pt]
\end{align}
Столбец свободных членов аппроксимируется в виде
\begin{equation}
    \label{eq:ns2d_bp}
    b^p[k(i+\tfrac12,j+\tfrac12)] = 
        -\left(
              \frac{u^*_{i+1,j+\tfrac12} - u^*_{i,j+\tfrac12}}{h_x}
            + \frac{v^*_{i+\tfrac12,j+1} - v^*_{i+\tfrac12,j}}{h_y}
        \right).
\end{equation}
Здесь используется функция перевода двумерного индекса в сквозной из $\eqref{eq:ns2d_kipjp}$.

В результате использования \eqref{eq:ns2d_du}, \eqref{eq:ns2d_dv} левая часть системы уравнений \eqref{eq:ns2d_pprime_slae}
будет постоянна на всех итерациях, что удобно для инициализации алгебраических решателей этой системы
(можно провести инициализацию один раз до начала счёта).

Это отличает эту систему от двух других систем, возникающих
из аппроксимации уравнений движения \eqref{eq:ns2d_ustar_slae}, \eqref{eq:ns2d_vstar_slae},
левые части которых зависят от значений с предыдущих
итерационных слоёв. Этот момент обуславливает выбор
решателей для этих систем, которые в эффективных гидродинамических кодах обычно
отличаются, от решателя для системы \eqref{eq:ns2d_pprime_slae}.


\subsubsection{Уравнение для поправки скорости}
И наконец рассмотрим аппроксимацию выражений \eqref{eq:ns2d_uprime_approx},
\eqref{eq:ns2d_vprime_approx}, которые примут явный вид
\begin{align}
    \label{eq:ns2d_uprime_discr}
    u'_{i,j+\tfrac12} = -d^u_{i,j+\tfrac12} \frac{p'_{i+\tfrac12,j+\tfrac12} - p'_{i-\tfrac12, j+\tfrac12}}{h_x}, \\[10pt]
    \label{eq:ns2d_vprime_discr}
    v'_{i+\tfrac12,j} = -d^v_{i+\tfrac12,j} \frac{p'_{i+\tfrac12,j+\tfrac12} - p'_{i+\tfrac12, j-\tfrac12}}{h_x}.
\end{align}

\subsubsection{Учёт граничных условий}
\label{sec:simple-bc}

Для уравнений Навье-Стокса на каждой границе расчётной области
требуется столько условий, сколько есть уравнений движения.
Для двумерной задачи \cref{eq:ns2d_u,eq:ns2d_div,eq:ns2d_v}
нужно задать два граничных условия.

При использовании разнесённой сетки граница области проходит
по граням основной сетки. 
На нижней и верхней границах расчётной области
присутствуют узлы для $v$, но отсутствуют
узлы для $u$.
На правой и левой границах, наоборот,
есть узлы с заданными компонентами $u$,
но нет узлов с компонентами $v$.
Узловые значения для давления $p$
никогда не бывают граничными.

Для простоты пока будем рассматривать только случай с заданными значениями
двух компонент скорости на каждой из границ задачи:
\begin{align*}
    &\left. u(x, y) \right|_{x,y\in\Gamma} = u^\Gamma(x, y), \\
    &\left. v(x, y) \right|_{x,y\in\Gamma} = v^\Gamma(x, y).
\end{align*}

В схеме SIMPLE частные граничные условия 
для скорости учитываются при решении задачи
для пробных скоростей $u^*, v^*$.
Тогда для поправки скорости $u', v'$ на границах
будут справедливы соответствующие однородные граничные условия (нулевые значения в нашем случае):

\begin{align}
    \label{eq:ns2d_usplit_bc}
    &\left. u^*(x, y) \right|_{x,y\in\Gamma} = u^\Gamma(x, y), \\
    \nonumber
    &\left. v^*(x, y) \right|_{x,y\in\Gamma} = v^\Gamma(x, y), \\
    \nonumber
    &\left. u'(x, y) \right|_{x,y\in\Gamma} = 0,\\
    \nonumber
    &\left. v'(x, y) \right|_{x,y\in\Gamma} = 0.
\end{align}

Для учёта граничных условий по скорости требуется модифицировать
системы линейных уравнений \eqref{eq:ns2d_ustar_slae}, \eqref{eq:ns2d_vstar_slae}.

Рассмотрим нижнюю границу $j=0$.

На нижней границе явно присутствуют узлы ``синей'' сетки.
Значит можно явно установить значения для скорости $v$
путём постановки нулей с единицой на диагонали в строке матрицы и отнесением необходимого
граничного значение в правый вектор столбец системы \eqref{eq:ns2d_vstar_slae}:
\begin{align}
    \label{eq:ns2d_bc1}
    &A^v[k(i+\tfrac12, 0), s] = \delta_{ks}, \quad \forall i, \; \forall s\\[10pt]
    \nonumber
    &b^{v}[k(i+\tfrac12, 0)] = v^\Gamma.
\end{align}
Такая модификация просто заменяет $k(i+\tfrac12, 0)$ -ое уравнение
системы \eqref{eq:ns2d_vstar_slae} на выражение
\begin{equation*}
    v^*_{i+\tfrac12, 0} = v^\Gamma.
\end{equation*}

Узлов для компонет $u$ на нижней границе нет.
Рассмотрим первый ряд точек ``красной'' сетки: $(i, \tfrac12)$.
Если бы мы захотели заполнить коэффициенты системы
линейных уравнений \eqref{eq:ns2d_ustar_slae}
по выведенным выше формулам \eqref{eq:ns2d_au}
для узла, расположенного в этом ряду, мы бы столкнулись
с необходимостью установки значения в фиктивную колонку:
последнее из уравнений \eqref{eq:ns2d_au}
предписывает нам установить значение по адресу
$[k(i, \tfrac12), k(i, -\tfrac12)]$, который, очевидно, не присутствует в матрице.

Действительно, $k(i,\tfrac12)$-ая строка системы уравнений \eqref{eq:ns2d_au}
имеет вид
\begin{equation}
    \label{eq:ns2d_au_low}
      D    u^*_{i, \tfrac12}
    + U^1  u^*_{i+1, \tfrac12}
    + L^1  u^*_{i-1, \tfrac12}
    + U^2 u^*_{i, \tfrac32}
    + L^2 u^*_{i, -\tfrac12} 
    = b^u_{i, \tfrac12},
\end{equation}
где $D$ -- коэффициент с основной диагонали, $U^{1,2}, L^{1,2}$ -- 
коэффициенты с двух верхних и двух нижних диагоналей, вычисляемые по формулам \eqref{eq:ns2d_au}.
Вторая нижняя диагональ у этой строки матрицы отсутствует.
Она соответствует вкладу от узла $(i, -\tfrac12)$, который лежит вне области расчёта, на полшага ниже
нижней границе.

Тем не менее, такой фиктивный узел мы можем 
использовать для записи аппроксимации
\begin{equation*}
    u^*_{i,0} = u^\Gamma = \frac{u^*_{i,\tfrac12} + u^*_{i, -\tfrac12}}{2} + o(h_x^2).
\end{equation*}
или
\begin{equation*}
    u^*_{i, -\tfrac12} \approx 2u^\Gamma - u^*_{i,\tfrac12}.
\end{equation*}

Подставляя это выражение в строку \eqref{eq:ns2d_au_low} получим
\begin{equation*}
      (D - L^2) u^*_{i, \tfrac12}
    + U^1       u^*_{i+1, \tfrac12}
    + L^1       u^*_{i-1, \tfrac12}
    + U^2       u^*_{i, \tfrac32}
    = b^u_{i, \tfrac12} + 2 u^\Gamma.
\end{equation*}

Таким образом, добавление коэффициента в фиктивную колонку строки матрицы при
наличие условия первого рода на границе равносильно
вычитанию этого коэффициента из диагонального элемента этой строки
и вычитанием удвоенного граничного значения из правой части.
В случае нижней границы получим
\begin{align}
    \label{eq:ns2d_bc2}
    &A^u[k(i, \tfrac12), k(i, \tfrac12)] \mathrel{-}= A^u[k(i, -\tfrac12)], \\[10pt]
    \nonumber
    &b^u[k(i, \tfrac12)] \mathrel{-}= 2 u^\Gamma.
\end{align}

Приёмы \eqref{eq:ns2d_bc1}, \eqref{eq:ns2d_bc2}
используются и на остальных границах для
постановки граничных условий для скорости.

При сборке системы линейных уравнений для
поправки давления \eqref{eq:ns2d_pprime_slae}
так же возникает проблема с обращением
к фиктивным узлам. Например, при рассмотрении левой стенки ($i=0$
третье из уравнений \eqref{eq:ns2d_ap} описывает
несуществующий столбец $k(-\tfrac12, j+\tfrac12)$.
Если обратиться к выражению \eqref{eq:ns2d_d2pdx2},
то будет видно, что это слагаемое пришло в результате
расписывания граничной производной $p'$,
которая, исходя из выражения \eqref{eq:ns2d_uprime_approx} пропорциональна граничному значению $u'$,
то есть, вспоминая \eqref{eq:ns2d_usplit_bc}, равна нулю:
\begin{equation*}
\left. \dfr{p'}{x} \right|_{0,j+\tfrac12} = -\frac{1}{\tau d^u} u'_{0,j+\tfrac12} = 0.
\end{equation*}

То есть добавлять слагаемые, соответствующие фиктивным узлам, в матрицу $A^p$ не нужно.
Не нарушая общности выведённых ранее выражений \eqref{eq:ns2d_ap},
просто модифицируем значения коэффициентов $d^u, d^v$:
\begin{align}
    \label{eq:ns2d_bc3}
    d^u_{0, j+\tfrac12} = d^u_{n_x+1, j+\tfrac12} = 0, \\[10pt]
    \nonumber
    d^v_{i+\tfrac12, 0} = d^u_{i+\tfrac12, n_y+1} = 0.
\end{align}

В исходных уравнениях
\eqref{eq:ns2d_u}-\eqref{eq:ns2d_div}
давление присутствует только в виде своих производных.
Если в задаче нигде не задано явное граничное условие
для давления, то решение для давления
будет определено только с точностью до константы.
Чтобы убрать эту неопределённость
рекомендуется явно положить давление нулю
в любом узле. Например, в случае нулевого узла,
по аналогии с \eqref{eq:ns2d_bc1} запишем:
\begin{align}
    \label{eq:ns2d_bc4}
    &A^p[k(\tfrac12, \tfrac12), s] = \delta_{ks}, \\[10pt]
    \nonumber
    &b^{p}[k(\tfrac12, \tfrac12)] = 0.
\end{align}


