\subsection{Метод конечных элементов}
\subsubsection{Формулировка}
Формулировка классического метода конечных элементов состоит из последовательного применения трёх методик:
\begin{itemize}
\item слабая (интегральная) постановка задачи с помощью метода взвешенных невязок,
\item использование метода Бубнова--Галёркина для записи интегральных соотношений для коэффициентов СЛАУ,
\item построение базисных функций с локальным носителем на основе конечноэлементной сетки.
\end{itemize}
\subsubsubsection{Метод взвешенных невязок}
TODO
\subsubsubsection{Метод Бубнова--Галёркина}
TODO
\paragraph{Пример с применением степенных базисных функций}
TODO
\subsubsubsection{Конечноэлементные базисные функции}
TODO
\subsubsection{Вывод СЛАУ для аппроксимации уравнения Пуассона}
\paragraph{Метод взвешенных невязок}
Исходное уравнение \cref{eq:poissonnd}
домножим на пробную функцию $q$ и проинтегрируем по
области решения
\begin{equation*}
-\arint{\nabla^2 u \, q}{\Omega}{\vec x} = \arint{f \, q}{\Omega}{\vec x}.
\end{equation*}
Чтобы сравнять порядок производной перед искомой функций $u$ и пробной
функцией $q$ применим формулу интегрирования по частям \cref{eq:partint_laplace_fg}:
\begin{equation*}
-\arint{\dfr{u}{n}\, q}{\partial\Omega}{s} + \arint{\nabla u \cdot \nabla q}{\Omega}{\vec x} = \arint{f \, q}{\Omega}{\vec x}.
\end{equation*}
Полученное выражение -- есть окончательная слабая постановка задачи.
\paragraph{Метод Бубнова--Галёркина}
В качестве пробной функции $q$ будем испльзовать набор базисных функций $\phi_i$, $i\in\overline{0, N-1}$.
По этому же базису разложим входящие в постановку скалярные функции $u$ и $f$.
Получим систему из $N$ линейных уравнений
\begin{equation*}
-\arint{\dfr{u}{n}\, \phi_i}{\partial\Omega}{s} +
\sum_{j=0}^{N-1}\left(\arint{\nabla \phi_j \cdot \nabla \phi_i}{\Omega}{\vec x}\right) u_j = \sum_{j=0}^{N-1}\left(\arint{\phi_j \, \phi_i}{\Omega}{\vec x}\right) f_j.
\end{equation*}
Первый из интегралов будет не равным нулю только для строк $i$, соответсвтующих
граничным узлам (базисам). Во всех остальных базисная функция 
согласно свойству согласованности будет равна нулю.
Тогда строки СЛАУ, соответсвующие внутренним узлам, примут вид
\begin{equation*}
\mat{S} u = \mat{b},
\quad \mat {b} = \mat M f,
\end{equation*}
где элементы матрицы масс $\mat M$ и матрицы жёсткости $\mat S$ будут равны
\begin{align}
\label{eq:fem_mass_mat}
&M_{ij} = \arint{\phi_j \phi_i}{\Omega}{\vec x},\\
\label{eq:fem_stiff_mat}
&S_{ij} = \arint{\nabla \phi_j \cdot \nabla \phi_i}{\Omega}{\vec x}.
\end{align}

Отдельно рассмотрим вычисление интеграла от некоторой функции $g$ в 
рамках аппроксимации Бубнова--Галёркина:
\begin{equation*}
\arint{g}{\Omega}{\vec x} = \sum_{j=0}^{N-1}V_i g_i,
\end{equation*}
где элементы ветора нагрузок $\mat V$ будут равны
\begin{equation}
\label{eq:fem_load_vec}
V_{i} = \arint{\phi_i}{\Omega}{\vec x}.
\end{equation}

\subsubsubsection{Одномерный пример}
TODO
\subsubsection{Техника сборки конечноэлементных векторов и матриц}
\subsubsubsection{Элементные вектора}
Распишем интеграл \cref{eq:fem_load_vec} по области $\Omega$
как сумму интегралов по отдельным элементам $E_m$.
При вследствии свойства локальности конечноэелементных базисов
в сумме можно оставить лишь элементы, инцидентные базису $i$:
\begin{equation*}
V_{i} = \arint{\phi_i}{\Omega}{\vec x} = \sum_{m \in \rm J^i} \arint{\phi^{(m)}_i}{E_m}{\vec x}.
\end{equation*}
Определим элементный вектор нагрузок $V^{(m)}$ 
как 
\begin{equation}
\label{eq:fem_elem_load_vector}
V^{(m)}_j = \arint{\phi^{(m)}_{g(j)}}{E_m}{\vec x}.
\end{equation}
Его размерность равна количеству степеней свободы элемента $N^{(m)}$.
Здесь запись $g(j)$ -- это перевод локального внутриэлементного индекса $j\in\overline{0, N^{(m)}-1}$
в глобальный индекс $i\in\overline{0, N}$.
\subsubsubsection{Элементные матрицы}
Аналогично можно расписать интегралы из \cref{eq:fem_mass_mat,eq:fem_stiff_mat}
через элементные интегралы. В частности для матрицы масс получим
\begin{equation*}
M_{ij} = \arint{\phi_i \phi_j}{\Omega}{\vec x} = \sum_{m \in \rm J^i} \arint{\phi^{(m)}_i \, \phi^{(m)}_j}{E_m}{\vec x}.
\end{equation*}
а выражение для локальной матрицы масс размерности $N^{(m)}\times N^{(m)}$ примет вид
\begin{equation}
\label{eq:fem_elem_mass_matrix}
M^{(m)}_{kl} = \arint{\phi^{(m)}_{g(k)} \, \phi^{(m)}_{g(l)}}{E_m}{\vec x}.
\end{equation}
По аналогии элементная матрица жёсткости запишется как
\begin{equation}
\label{eq:fem_elem_stiff_matrix}
S^{(m)}_{kl} = \arint{\nabla \phi^{(m)}_{g(k)} \cdot \nabla \phi^{(m)}_{g(l)}}{E_m}{\vec x}.
\end{equation}

\subsubsubsection{Алгоритм сборки}
После того, как элементные вектора/матрицы для
искомого интеграла определены, следует применить алгоритм
элементной сборки. Для этого понадобится
таблица связность \quo{элемент-индексы базисов}.
В простейшем случае линейных лагранжевых базисов
эта таблица эквивалентна геометрической таблице \quo{ячейка-узлы}.
Назовём эту таблицу $glob$. Тогда псеводкод для сборки вектора примет вид
\begin{equation}
\label{eq:fem_vector_assemble}
\begin{array}{ll}
V_i = 0                                                      & \textrm{-- инициализируем глобальный вектор нулями}\\
\textbf{for } m = \overline{0, N-1}                          & \textrm{-- цикл по конечным элементам}\\
\qquad N^m = dof(m)                                                   & \textrm{-- количество степеней свободы у элемента}\\ 
\qquad \textbf{for } i = \overline{0, N^m-1}              & \textrm{-- цикл по базисам внутри элемента}\\
\qquad \qquad g = glob(m, i)                                & \textrm{-- глобальный индекс локального базиса}\\
\qquad \qquad V_{g} \pluseq  V^{m}_i                              & \\
\qquad \textbf{endfor}                                       & \\
\textbf{endfor}
\end{array}
\end{equation}
В аналогичном алгоритме для матрицы добавится ещё один цикл:
\begin{equation}
\label{eq:fem_matrix_assemble}
\begin{array}{ll}
M_{ij} = 0                                                      & \textrm{-- инициализируем глобальную матрицу нулями}\\
\textbf{for } m = \overline{0, N-1}                          & \textrm{-- цикл по конечным элементам}\\
\qquad N^m = dof(m)                                                   & \textrm{-- количество степеней свободы у элемента}\\ 
\qquad \textbf{for } i = \overline{0, N^m-1}              & \textrm{-- цикл по базисам внутри элемента}\\
\qquad \qquad \textbf{for } j = \overline{0, N^m-1}              & \textrm{-- цикл по базисам внутри элемента}\\
\qquad \qquad \qquad g = glob(m, i)                                & \textrm{-- глобальный индекс локального базиса}\\
\qquad \qquad \qquad h = glob(m, j)                               & \\
\qquad \qquad \qquad M_{gh} \pluseq  M^{m}_{ij}                              & \\
\qquad \qquad \textbf{endfor}                                       & \\
\qquad \textbf{endfor}                                       & \\
\textbf{endfor}
\end{array}
\end{equation}
\subsubsection{Вычисление элементных интегралов в модельном пространстве}
Будем вычислять элементные интегралы \cref{eq:fem_elem_mass_matrix,eq:fem_elem_stiff_matrix,eq:fem_elem_load_vector}
в модельном пространстве $\vec \xi$.
Для этого введём преобразование координат $\vec x \rightarrow \vec \xi$ согласно п.\ref{sec:coo_transform}.
Интеграл для определения локальной матрицы масс \cref{eq:fem_elem_mass_matrix} 
в модельных координатах распишется согласно формуле \cref{eq:dxideta_integral}
\begin{equation}
\label{eq:fem_mass_matrix_xi}
M^{(m)}_{ij} = \arint{N^{(m)}_i N^{(m)}_j|J^{(m)}|}{\tilde E_m}{\vec \xi}
\end{equation}
Здесь $\tilde E_m$ -- параметрический образ конечного элемента $E^k$,
$|J^{(m)}(\vec \xi)|$ - Якобиан преобразования для $m$-ого элемента,
$N^{(m)}_i(\vec \xi)$ -- функция формы, или часть базисной фукнции, заданная в $m$-ом элементе в модельных координатах.
То есть
\begin{equation*}
N_i^{(m)}(\vec \xi) = \phi_{g(i)}^{(m)}(\vec x(\vec \xi)).
\end{equation*}
Локальный вектор нагрузок записывается из соотношения \cref{eq:fem_elem_load_vector}:
\begin{equation}
\label{eq:fem_load_vector_xi}
V^{(m)}_{i} = \arint{N^{(m)}_i |J^{(m)}|}{\tilde E_m}{\vec \xi}
\end{equation}
Локальная матрица жёсткости из \cref{eq:fem_elem_stiff_matrix}:
\begin{equation}
\label{eq:fem_stiff_matrix_xi}
S^{(m)}_{ij} = \arint{\nabla_{\vec x} N^{(m)}_i \cdot \nabla_{\vec x} N^{(m)}_j |J^{(m)}|}{\tilde E_m}{\vec \xi}
\end{equation}
Здесь $\nabla_{\vec x} N^{m}_i$ -- градиент shape-функции (заданного в модельном пространстве) по физическим координатам.
Для его вычисления следует воспользоваться формулами \cref{eq:vec_grad_dx}

%\subsubsubsection{Треугольный элемент. Линейный двумерный базис}
%Матрица масс (из \cref{eq:mass_matrix}):
%\begin{equation}
%\label{eq:mass_matrix_lintri}
%M^E_{ij} = \int\limits_0^1 \int\limits_0^{1-\xi} \phi_i(\xi, \eta) \phi_j(\xi, \eta) |J| \, d\eta d\xi =
%\frac{|J|}{24}\left(
%\begin{array}{ccc}
%2 & 1 & 1 \\
%1 & 2 & 1 \\
%1 & 1 & 2
%\end{array}
%\right)
%\end{equation}
