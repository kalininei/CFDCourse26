\subsection{Лекция 4 (04.10.25) Непостоянный коэффициент диффузии, учёт особенностей}
Теория 
\secref{sec:fvm_nonconst_lambda},
\secref{sec:fvm_log_singularity},
\secref{sec:fvm_radsym}.

В тесте \cvar{poisson1-fvm-radial} из файла \ename{poisson_fvm_test.cpp}
реализовано решение одномерного уравнения Пуассона с граничными условиями первого рода и неоднородным коэффициентом диффузии в радиально-симметричной постановке
\begin{align*}
&\frac1r\dfr{}{r}\left(\lambda(r) r\dfr{u}{r}\right) = 0, \\
&u(r_0) = a, \quad u(r_1) = b,\\
&\lambda(r) = \begin{cases}
\lambda_0, & r < r_k,\\
\lambda_1, & r \geq r_k.\\
\end{cases}
\end{align*}
Общий вид решения этого уравнения примет вид
\begin{equation*}
u(r) = 
\begin{cases}
u_{0} = A_0 \ln(r) + B_0, & r < r_k\\
u_{1} = A_1 \ln(r) + B_1, & r \geq r_k.
\end{cases}
\end{equation*}
Четыре неизвестные константы находятся из двух граничных условий плюс двух условий на границе скачка коэффициента диффузии:
\begin{equation}
\label{eq:fvm_radial_partial_lambda_conditions}
\begin{array}{l}
r = r_0: \quad u_0 = a;\\
r = r_1: \quad u_1 = b;\\
r = r_k: \quad u_0 = u_1, \quad \dsfr{u_0}{r} = \dsfr{u_1}{r}.
\end{array}
\end{equation}
Использовались параметры
\begin{equation*}
r_0 = 0.05, \quad r_1 = r_0 + 1.0, \quad r_k = r_0 + 0.5, \quad a = 1, \quad b = 0, \quad \lambda_0 = 10, \quad \lambda_1 = 1.
\end{equation*}
Для определения коэффициента диффузии на границе использовалось среднее арфифметическое \cref{eq:fvm_lambda_ij_0}.
Учёт логарифмической особенности не производился.
Необходимо:
\begin{enumerate}
\item Изменить способ вычисления коэффициента диффузии $\lambda_{ij}$ на границе $\gamma_{ij}$ на формулу \cref{eq:fvm_lambda_ij_2},
\item Добавить учёт логарифмической особенности при вычислении нормальной производной около внутренней границы
\item Сравнить полученные решения на графиках для грубых сеток.
      Построить графики сеточной сходимости и сравнить порядки аппроксимации: первоначальной схемы, схемы с улучшением из п.1, схемы с улучшением из п.2 и схемы с обоими улучшениями.
\item Решить ту же задачу в декартовой 2D постановке на неструктурированной сетке с итерационной поправкой на неортогональность.
      Использовать учёт логарифмической особенности.
      Сравнить сеточную сходимость в случае использования формул \cref{eq:fvm_lambda_ij_1} и \cref{eq:fvm_lambda_ij_2}.
\item Решить аналогичную сферически-симметричную задачу с учётом сферической особенности. Показать второй порядок аппроксимации решения.
\end{enumerate}

Как было показано в \secref{sec:fvm_radsym},
решение в конечнообъёмная схема в радиально-симметричном случае
отличается от расчётной схемы в декартовых координатах только формулой вычисления
мер площади и объёмов.
Поэтому для этого случая был создан отдельный класс радиальных сеток \cvar{RadialGrid1D},
в котором формулы вычисления площадей были переписаны согласно формулам \cref{eq:fvm_radsym_measures}.

Коэффициент диффузии в точках коллокации хранится в поле \cvar{lambda_}.
Вычисление его значения на грани $\lambda_{ij}$ происходит в методе \cvar{face_lambda}.

Для выполнения пункта 5 нужно будет получить точное решение для задачи
\begin{align*}
&\frac1{r^2}\dfr{}{r}\left(\lambda(r) r^2\dfr{u}{r}\right) = 0, \\
&u(r_0) = a, \quad u(r_1) = b,\\
&\lambda(r) = \begin{cases}
\lambda_0, & r < r_k,\\
\lambda_1, & r \geq r_k.\\
\end{cases}
\end{align*}
Для этого нужно использовать те же условия сращивания
\cref{eq:fvm_radial_partial_lambda_conditions},
но с общим решением вида
\begin{equation*}
u(r) = 
\begin{cases}
u_{0} = \dfrac{A_0}r + B_0, & r < r_k\\[10pt]
u_{1} = \dfrac{A_1}r + B_1, & r \geq r_k.
\end{cases}
\end{equation*}

\paragraph{Рекомендации к программированию}
Для учёта особенности (п.~2) нужно модифицировать
коэффициент диффузии для граничных граней на границе $r_0$
по формуле \cref{eq:fvm_ln_singularity_hij} в методе \cvar{face_lambda}.

Для выполнени пункта 4 понадобится построить cетку в плоскости $(x, y)$. 
Для этого следует использовать сетку из построителя \ename{radial.py},

Для пункта 5 будет удобно по аналогии с классом \cvar{RadialGrid1D}
создать класс сеток \cvar{SphericalGrid1D} с соответствующими правилами вычисления площадей и объёмов.
