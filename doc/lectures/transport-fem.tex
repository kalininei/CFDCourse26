\subsection{МКЭ}
\subsubsection{Линейные схемы}
Полудискретизованная схема для уравнения переноса имеет вид
\begin{equation}
\label{eq:fem_transport_semidis}
\mat M \dfr{u}{t} = \mat K u
\end{equation}
\subsubsection{Метод коррекции потока FCT}
Перепишем уравнение
\cref{eq:fem_transport_semidis}
и использованием операторов низкого порядка
\begin{equation}
\label{eq:fem_fct_low}
\mat M_L \dfr{u}{t} = \mat L^\theta u + f
\end{equation}
Здесь антидиффузное слагаемое $f$ компенсирует все погрешности, которые вносят операторы низкого порядка.
Оператор переноса в двухслойной схеме примет вид
\begin{equation*}
L^\theta u = \theta L \hat u + (1 - \theta) L u.
\end{equation*}

Вычтем \cref{eq:fem_transport_semidis} из 
\label{eq:fem_fct_low}
и выразим эту анидифузию. Получим
\begin{equation}
\label{eq:fem_fct_fij}
f_{ij} = m_{ij}\left(\left(\dfr{u}{t}\right)_j   - \left(\dfr{u}{t}\right)_i \right) - d^\theta_{ij} (u_j - u_i).
\end{equation}
В двухлойных схемах производные аппроксимируются в виде
\begin{equation*}
\dfr{u}{t} \approx \frac{\hat u - u}{\dt}
\end{equation*}
а оператор диффузии как
\begin{equation*}
d_{ij}^\theta u_i = \theta d_{ij}\hat u_i + (1 - \theta) d_{ij} u_i
\end{equation*}

Добавление полной антидиффузии провоцирует осцилляции в чиленном решении.
Чтобы их срезать, антидиффузию ограничивают:
\begin{equation}
f^*_{ij} = \alpha_{ij} f_{ij}
\end{equation}
Параметр $\alpha_ij$ подбирают таким образом, что добавка на спровоцировала
локальный экстремум. Для постановки ограничений используют
промежуточное монотонное решение 
\begin{equation}
\label{eq:fem_fct_utilde}
\mat M_L \frac{\tilde u - u}{\dt} = (1 - \theta) \mat L u
\end{equation}
которое относится к моменту времени $t_0 + \theta \dt$.

Итерационный процесс на временном слое:
\begin{enumerate}
\item
Из уравнения \cref{eq:fem_fct_utilde} найти промежуточное решение
\item
Задать начальное приближение решения на слое. Например $\hat u = \tilde u$.
\item
Из уравнений \cref{eq:fem_fct_fij} вычислить полную антидиффузию
\item Используя найденную антидиффузию и решение $\tilde u$ вычислить ограничения $\alpha_{ij}$ и ограниченные поток $f^*$
\item Подсчитать невязку с найденной новой антидиффузиуй. Выйти, если она достаточно мала
\item Решить систему \cref{eq:fem_fct_low} используя найденный поток $f^*$.
\item Перейти на п.3
\end{enumerate}

Это итерационный процесс можно реализовать в терминах коррекции поправки
с предобуславливателем $\mat M_L - \theta \dt/2 L$.

\subsubsubsection{Линеаризованный FCT}
Можно избавится от итерационного процесса, если 
аппроксимировать производные через найденное значение функции на
промежуточном слое $\tilde u$:
\begin{equation}
\label{eq:fem_fct_dudt_linearized}
\dfr{u}{t} \approx \frac{\tilde u - u}{(1 - \theta)\dt}
\end{equation}
