\subsection{МКЭ}
\subsubsection{Линейные схемы}
Полудискретизованная схема для уравнения переноса имеет вид
\begin{equation}
\label{eq:fem_transport_semidis}
\mat M \dfr{u}{t} = \mat K u
\end{equation}

\subsubsubsection{Противопотоковый оператор переноса}
\begin{equation*}
\mat L = \mat K + \mat D
\end{equation*}
\begin{equation}
\label{eq:fem_transport_dij}
d_{ij} = \begin{cases}
\max(0, -k_{ij}, -k_{ji}),  &\quad i\neq j \\
-\sum_{k \neq i} d_{ik},    &\quad i = j
\end{cases}
\end{equation}

\subsubsection{Метод коррекции потока FCT}
Перепишем уравнение
\cref{eq:fem_transport_semidis}
и использованием операторов низкого порядка
\begin{equation}
\label{eq:fem_fct_low}
\mat M_L \dfr{u}{t} = \mat L^\theta u + f
\end{equation}
Здесь антидиффузное слагаемое $f$ компенсирует все погрешности, которые вносят операторы низкого порядка.
Оператор переноса в двухслойной схеме примет вид
\begin{equation*}
L^\theta u = \theta L \hat u + (1 - \theta) L u.
\end{equation*}

Вычтем \cref{eq:fem_transport_semidis} из 
\label{eq:fem_fct_low}
и выразим эту анидифузию. Получим
\begin{equation}
\label{eq:fem_fct_fij}
f_{ij} = m_{ij}\left(\left(\dfr{u}{t}\right)_j   - \left(\dfr{u}{t}\right)_i \right) - d^\theta_{ij} (u_j - u_i).
\end{equation}
В двухлойных схемах производные аппроксимируются в виде
\begin{equation*}
\dfr{u}{t} \approx \frac{\hat u - u}{\dt}
\end{equation*}
а оператор диффузии как
\begin{equation*}
d_{ij}^\theta u_i = \theta d_{ij}\hat u_i + (1 - \theta) d_{ij} u_i
\end{equation*}

Добавление полной антидиффузии провоцирует осцилляции в чиленном решении.
Чтобы их срезать, антидиффузию ограничивают:
\begin{equation}
f^*_{ij} = \alpha_{ij} f_{ij}
\end{equation}
Параметр $\alpha_ij$ подбирают таким образом, что добавка на спровоцировала
локальный экстремум. Для постановки ограничений используют
промежуточное монотонное решение 
\begin{equation}
\label{eq:fem_fct_utilde}
\mat M_L \frac{\tilde u - u}{\dt} = (1 - \theta) \mat L u
\end{equation}
которое относится к моменту времени $t_0 + \theta \dt$.

Итерационный процесс на временном слое:
\begin{enumerate}
\item
Из уравнения \cref{eq:fem_fct_utilde} найти промежуточное решение
\item
Задать начальное приближение решения на слое. Например $\hat u = \tilde u$.
\item
Из уравнений \cref{eq:fem_fct_fij} вычислить полную антидиффузию
\item Используя найденную антидиффузию и решение $\tilde u$ вычислить ограничения $\alpha_{ij}$ и ограниченные поток $f^*$
\item Подсчитать невязку с найденной новой антидиффузиуй. Выйти, если она достаточно мала
\item Решить систему \cref{eq:fem_fct_low} используя найденный поток $f^*$.
\item Перейти на п.3
\end{enumerate}

Это итерационный процесс можно реализовать в терминах коррекции поправки
с предобуславливателем $\mat M_L - \theta \dt/2 L$.

\subsubsubsection{Линеаризованный FCT}
Можно избавится от итерационного процесса, если 
аппроксимировать производные через найденное значение функции на
промежуточном слое $\tilde u$:
\begin{equation}
\label{eq:fem_fct_dudt_linearized}
\dfr{u}{t} \approx \frac{\tilde u - u}{(1 - \theta)\dt}
\end{equation}

\subsubsection{FEM-TVD}
\paragraph{Идея ограничения потока:} добавить к оператору переноса первого порядка $\mat L = \mat K + \mat D$
антидиффузию,  при этом не спровоцировов появление осцилляций в решении:
\begin{equation}
\mat M_L\dfr{u}{t} = \mat K^*[u] u, \quad  \mat K^*[u] = \mat K + \mat D - \mat F[u]
\end{equation}
В случае условия несжимаемости для скорости переноса можно записать
\begin{equation}
\label{eq:fem_tvd_init}
m_i\dfr{u_i}{t} = \sum_{j\neq i} \left(k_{ij} + d_{ij} - f_{ij}\right) (u_j - u_i) =
\sum_{j\neq i} k^*_{ij} (u_j - u_i).
\end{equation}

\paragraph{Одномерный случай}
\begin{align*}
&\dfr{u_i}{t} + U\frac{u_{i+1} - u_{i-1}}{2h} = 0, \quad \text{схема второго порядка} \\
&\dfr{u_i}{t} + U\frac{u_{i} - u_{i-1}}{h} = 0      \quad \text{схема первого порядка}
\end{align*}
Тогда
\begin{align*}
&\dfr{u_i}{t} = \frac{U}{2h}(u_{i-1} - u_i) - \frac{U}{2h}(u_{i+1} - u_i), \\
&\dfr{u_i}{t} = \frac{U}{h}(u_{i-1} - u_i)
\end{align*}
Элементы матриц $\mat K$, $\mat L$, $\mat D$:
\begin{equation}
\label{eq:fem_tvd_matrices_1d}
\begin{array}{rcccl}
& i-1 & i & i+1 & \\
\hline
(\mat K)_i = [ & \frac{U}{2h}  & 0 & -\frac{U}{2h} & ] \\
(\mat L)_i = [ & \frac{U}{h}  & -\frac{U}{h} & 0 & ] \\
(\mat D)_i = [ & \frac{U}{2h}  & -\frac{U}{h} & \frac{1}{2h} & ] \\
\end{array}
\end{equation}

\paragraph{Дискретизация и устойчивость}
\begin{equation}
\label{eq:fem_tvd_tau_condition}
\dt \leq -\frac{m_i}{(1-\theta)k^*_{ii}}, \quad k^*_{ii} = l_{ii} - f_{ii}.
\end{equation}

\paragraph{Идея TVD-схем} -- введём критерий экстремума $r_{ij}$ такой что
\begin{equation}
f_{ij} = \Phi(r_{ij}) d_{ij} > 0.
\end{equation}

$F[u]$ -- симметричный оператор $\hence f_{ij} = f_{ji}$.
Рассмотрим направленную связь $\overrightarrow{ij}$, такую что $k_{ij} < 0$.
Тогда:
\begin{align*}
k_{ij} < 0, \\
k_{ji} > 0, \\
l_{ij} = 0, \\
l_{ji} > 0.
\end{align*}
Для выполнения критерия Хартена уравнения
\cref{eq:fem_tvd_init}
в $j$-ой строке достаточно положить
\begin{equation}
\label{eq:fem_tvd_fji}
f_{ji} = f_{ij} < k_{ji} + d_{ji} = l_{ji}.
\end{equation}
Теперь рассмотрим строку $i$.
Очевидно, что формальный критерий Хартена $k^*_{ij} \geq 0$ не выполняется:
\begin{equation}
k^*_{ij} = -f_{ij} = -\Phi(r_{ij}) d_{ij} < 0
\end{equation}
Выберем $\Phi(r_{ij})$ таким образом, чтобы антидиффузный поток из узла $j$ в узел $i$
можно было разложить в сумму диффузных:
\begin{equation}
k^*_{ij} (u_j - u_i) = -f_{ij} (u_j - u_i) = \sum_{k\neq i} f^*_{ijk}(u_k - u_i)
\end{equation}

Во первых зададим симметричность функции ограничителя:
\begin{equation}
\Phi(r) = r \Phi(1/r)
\end{equation}
Тогда
\begin{equation}
-f_{ij}(u_j - u_i) = -\Phi(r_{ij}) (u_j - u_i) = -\Phi(1/r_{ij}) r_{ij} d_{ij} (u_j - u_i)
\end{equation}
Задача -- выразить $\Delta u_{ij}$ через $r^*_{ijk} \geq 0$
\begin{equation}
\label{eq:fem_tvd_rijk}
\Delta u_{ij} = -r_{ij}(u_j - u_i) = \sum_{k \neq i} r^*_{ijk}(u_k - u_i)
\end{equation}

\subsubsubsection{Интерполяционные методы}

В одномерном случае $j=i+1$ -- узел по потоку. 
\begin{equation}
r_{ij} = r_{i,i+1} = \frac{u_i - u_{i-1}}{u_{i+1} - u_{i}}
\hence
\Delta u_{i,i+1} = u_{i-1} - u_{i}.
\end{equation}
Многомерным обобщением является
\begin{equation}
\label{eq:fem_tvd_rij_nd}
r_{ij} = \frac{u_i - u^*}{u_j - u_{i}}
\end{equation}

Где  $u^*$ -- значение в противопотоковом узле $\vec x^*$ (см. \figref{fig:fem_tvd_interpolation}) неизвестно
и подлежит определению.

\begin{figure}[h!]
\centering
\includegraphics[width=0.7\linewidth]{fem_tvd_interpolation.pdf}
\caption{Вспомогательный узел $\vec x^*$ на конечноэлементной сетке}
\label{fig:fem_tvd_interpolation}
\end{figure}

\paragraph{Интерполяция/Экстраполяция в инцидентном элементе $E_a$}
Среди элементов, инцидентрых узлу $i$, найдём элемент $E_a$, ближайший к фиктивной точке $\vec x^*$.
Он может как содержать в себе эту фиктивную точку, так и не содержать (на \figref{fig:fem_tvd_interpolation} представлен второй случай).
Тогда можно записать
\begin{equation}
\label{eq:fem_tvd_interpolation_ea}
u_i - u(\vec x^*) = \sum_{k\in E_a} \phi^{(a)}_k(\vec x^*) (u_i - u_k).
\end{equation}
Подставляя это выражение в \cref{eq:fem_tvd_rij_nd} и далее в \cref{eq:fem_tvd_rijk} получим
\begin{equation*}
r^*_{ijk} = \phi_k^{(a)}(\vec x^*).
\end{equation*}
Для линейных конечных элементов значения элементных базисных функций справделиво:
\begin{equation*}
\phi^{(a)}(\vec x) = \begin{cases}
\in [0, 1],  &\quad \vec x \in E_a, \\
\notin [0, 1],  &\quad \vec x \notin E_a.
\end{cases}
\end{equation*}
Поэтому в случае экстраполяции возможно $r^*_{ijk} < 0$, поэтому такая схема не является монотонной по критерию Хартена.

\paragraph{Интерполяция в содержащем элементе $E_b$}
Теперь для интерполяции используем элемент, точно содержащий $\vec x^*$
независимо от того, является ли он инцидентным узлу $i$. Тогда аналогично
\begin{equation}
\label{eq:fem_tvd_interpolation_eb}
u_i - u(\vec x^*) = \sum_{k\in E_b} \phi^{(a)}_k(\vec x^*) (u_i - u_k).
\end{equation}
В этом случае в линейных элементов критерий Хартена удовлетворяется. Но терятся локальность схемы.

\paragraph{Использование градиента в узле $u_i$}
Воспользуемся методикой восстановления фиктивного значения $u^*$
из градиента, ранее рассмотренного для метода конечных объёмов \secref{sec:grad_inteprolate_up}.
Выделим ось $\tau$, последовательно содержащую узлы $\vec x^*$, $\vec x_i$, $\vec x_j$ на равном удалении.
Рассмотрим производную по этой оси в центральном узле $i$.
С одной стороны из односторонней разности
\begin{equation*}
\left.\dfr{u}{\tau}\right|_i = \frac{u_i - u^*}{|\vec x_i - \vec x^*|}.
\end{equation*}
С другой стороны согласно проекии общего вектора градиента
\begin{equation*}
\left.\dfr{u}{\tau}\right|_i = \left( \nabla u \right)_i \cdot \frac{\vec x_j - \vec x_i} {|\vec x_j - \vec x_i| }.
\end{equation*}
Пользуясь равноудалённостью узлов запишем искомую величину как
\begin{equation*}
u_i - u^* = \left( \nabla u \right)_i \cdot (\vec x_j - \vec x_i).
\end{equation*}
Далее определим градиент в узле с использованием МКЭ подхода.
Запишем уравнение.
\begin{equation*}
\vec g = \nabla u
\end{equation*}
Далее домножим на пробную функцию и разложим сеточные функции по базису. Тогда, с использованием
сосредоточенной матрицы масс получим
\begin{equation*}
\vec g_i = \frac{1}{m_i}\sum_k \vec c_{ik} u_k,
\end{equation*}
где градиентная матрица вычисляется по формуле
\begin{equation*}
\vec c_{ik} = \arint{\nabla \phi_k \, \phi_i}{\Omega}{\vec x}
\end{equation*}
Используя свойство согласованности базисных функций $\sum_j \phi_j = 1$
и формулу интегрирования по частям
\cref{eq:partint}
несложно получить два свойства
этой матрицы:
\begin{align}
& \sum_k \vec c_{ik} = 0,                                                  \label{eq:fem_tvd_cij_prop1}\\
& \vec c_{ik} = -\vec c_{ki}, \quad \text{для внутренних узлов $i$, $k$}. \label{eq:fem_tvd_cij_prop2}
\end{align}
С помощью первого из этих свойств \cref{eq:fem_tvd_cij_prop1}
можно записать искомое выражение
\begin{equation}
\label{eq:fem_tvd_interpolation_gradient}
u_i - u^* = \frac{1}{m_i}\sum_{k\neq i} \vec c_{ik} \cdot (\vec x_j - \vec x_i) (u_k - u_i).
\end{equation}
Откуда 
\begin{equation}
\label{eq:fem_tvd_gradient_rijk}
r^*_{ijk} = \frac{1}{m_i}\vec c_{ik} \cdot (\vec x_j - \vec x_i).
\end{equation}
Из свойства антисимметричности матрицы \cref{eq:fem_tvd_cij_prop2}
следует, что $\vec c_{ik}$ могут быть отрицательными.
Так же отрицательными могут быть компоненты вектора разности $\vec x_j - \vec x_i$.
Поэтому схема \cref{eq:fem_tvd_interpolation_gradient}
не удовлетворяет критерию монотонности.

\paragraph{Использование противопоточного градиента в узле $u_i$}
Для того, чтобы вписаться в требование монотоноости модифицируем
выражение \cref{eq:fem_tvd_interpolation_gradient},
оставив в нём только положительные коэффициенты.
Чтобы компенсировать отброшенные слагаемые умножим
полученное выражение на два:
\begin{equation}
\label{eq:fem_tvd_interpolation_upwind_gradient}
u_i - u^* = \frac{2}{m_i}\sum_{k\neq i} \max(0, \vec c_{ik} \cdot (\vec x_j - \vec x_i)) (u_k - u_i).
\end{equation}
\begin{equation}
Тогда
\label{eq:fem_tvd_upwind_gradient_rijk}
r^*_{ijk} = \frac{2}{m_i}\max(0, \vec c_{ik} \cdot (\vec x_j - \vec x_i)).
\end{equation}
Этот подход к построению противопотокового градиента
похож на построение оператора переноса
первого порядка точности с помощью алгебраической методики \cref{eq:fem_transport_dij} (см. схему 
\cref{eq:fem_tvd_matrices_1d} для одномерной аналогии).

\subsubsubsection{Алгебраический метод}
Запишем одномерный критерий TVD в виде:
\begin{equation}
r_{i,i+1} = \frac{k_{i,i-1} (u_{i-1} - u_{i})}{k_{i,i+1} (u_{i+1} - u_{i})}
\end{equation}

Наивное многомерное обобщение могло бы выглядеть так
\begin{equation}
r_{ij} = \frac
{\displaystyle\sum_k \max(0, k_{ik}) (u_{k} - u_{i})}
{\displaystyle\sum_k \min(0, k_{ik}) (u_{k} - u_{i})}
\end{equation}

Тогда
\begin{equation}
r^*_{ijk} = \frac{-\max(0, k_{ik}) (u_j - u_i)} {\displaystyle\sum_k \min(0, k_{ik}) (u_k - u_i)}
\end{equation}

Такое определение не гарантирует положительность $r^*_{ijk}$.
Необходимо, чтобы знаки $(u_j - u_i)$ и $(u_k - u_i)$ совпадали.
Заметим ещё одно свойство одномерной схемы:
в случае монотонного поведения $u_i$, знаки разностей $(u_{i+1} - u_i)$, $(u_{i-1} - u_i)$ в числителе и знаменателе различаются.
Причём знак в знаменателе совпадает со знаком текущего узла по потоку $j$ (который в одномерном случае равен $i+1$).
Тогда естественным обобщением выглядит
\begin{equation}
r_{ij} = \begin{cases}
\frac {\displaystyle\sum_k \max(0, k_{ik}) \min(0, (u_{k} - u_{i}))} {\displaystyle\sum_k \min(0, k_{ik}) \min(0, (u_{k} - u_{i}))}, &\quad u_i \geq u_j, \\[30pt]
\frac {\displaystyle\sum_k \max(0, k_{ik}) \max(0, (u_{k} - u_{i}))} {\displaystyle\sum_k \min(0, k_{ik}) \max(0, (u_{k} - u_{i}))}, &\quad u_i < u_j.
\end{cases}
\end{equation}

Окончательно запишем
\begin{equation}
\label{eq:fem_tvd_algebraic_rij}
r_{ij} = \begin{cases}
Q_i^+ / P_i^+, &\quad u_i \geq u_j,\\
Q_i^- / P_i^-, &\quad u_i \geq u_i,\\
\end{cases}
\end{equation}

\begin{align}
&Q_i^{\pm} = \sum_k \max(0, k_{ik})\substack{\max \\ \min} (u_k - u_i), \\
&P_i^{\pm} = \sum_k \min(0, k_{ik})\substack{\min \\ \max} (u_k - u_i).
\end{align}

Вспомитная строку $j$
\cref{eq:fem_tvd_fji}
окончательно запишем антидиффузию:
\begin{equation}
f_{ij} = f_{ji} = \min(\Phi(r_{ij}) d_{ij}, l_{ji})
\end{equation}

Алгоритм будет иметь такой вид
\begin{enumerate}
\item
Вычислить $k_{ij}$ -- сеточную матрицу переноса второго порядка,
\item
Вычислить линейную диффузию $d_{ij} = \max(0, -k_{ij}, -k_{ji})$.
\item
Вычислить $l_{ij} = k_{ij} + d_{ij}$
\item
Для каждого узла $i$ определить $Q^{\pm}_i$, $P^{\mp}_i$
\item
В цикле для напраленных граней $\overrightarrow{ij}$ определить $f_{ij}$
\item
окончательно собрать матрицу переноса с ограничением
$k^*_{ij} = k_{ij} + d_{ij} - f_{ij}$.
\end{enumerate}
