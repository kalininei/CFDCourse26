\subsection{Hybmesh}
\label{sec:hybmesh}

Генератор сеток на основе композитного подхода.
Работает на основе python-скрипотов.
Полная документация \url{http://kalininei.github.io/HybMesh/index.html}

\subsubsection{Работа в Windows}
Инсталлятор программы следует скачать по ссылке
\url{https://github.com/kalininei/HybMesh/releases}
и установить стандартным образом.

Для запуска скрипта построения \ename{script.py} нужно
открыть консоль, перейти в папку с нужным скриптом,
оттуда выполнить (при условии, что программа была установлена в папку \ename{C:\Program Files}):
\begin{shelloutput}
> "C:\Program Files\HybMesh\bin\hybmesh.exe" -sx script.py
\end{shelloutput}

\subsubsection{Работа в Linux}
Версию для линукса нужно собирать из исходников.
Либо, если собрать не получилось,
можно строить сетки в Windows и переносить
полученные vtk-файлы на рабочую систему. 

Перед сборкой в систему необходимо установить dev-версии
пакетов \ename{suitesparse} и \ename{libxml2}. Также 
должны быть доступны компилляторы \ename{gcc-c++} и \ename{gcc-fortan} и \ename{cmake}.
Программа работает со скиптами python2.
Лучше установить среду anaconda (\url{https://docs.anaconda.com/free/anaconda/install/index.html})
И в ней создать окружение c python-2.7:
\begin{shelloutput}
> conda create -n py27 python=2.7   # создать среду с именем py27
> conda activate py27               # активировать среду py27
> pip install decorator             # установить пакет decorator
\end{shelloutput}

Сначала следует склонировать репозиторий в папку с репозиториями гита:
\begin{shelloutput}
> cd D:/git_repos
> git clone https://github.com/kalininei/HybMesh
\end{shelloutput}

Поскольку программа не предназначена для запуска из под анаконды,
в сборочные скрипты нужно внести некоторые изменения.
В корневом сборочном файле \ename{HybMesh/CMakeLists.txt} 
нужно закомментировать все строки в диапазоне
\begin{minted}[linenos=false]{text}
# ========================== Python check
....
# ========================== Windows installer options
\end{minted}
а в файле \ename{HybMesh/src/CMakeLists.txt} последнюю строку
\begin{minted}[linenos=false]{text}
#add_subdirectory(bindings)
\end{minted}

Далее, находясь в корневой директории репозитория HybMesh, запустить сборку
\begin{shelloutput}
> mkdir build
> cd build
> cmake .. -DCMAKE_BUILD_TYPE=Release
> make -j8
> sudo make install
\end{shelloutput}

Для запуска скриптов нужно создать скрипт-прокладку
\begin{minted}[linenos=false]{python}
import sys
sys.path.append("/path/to/HybMesh/src/py/")  # вставить полный путь к Hybmesh/src/py
execfile(sys.argv[1])
\end{minted}
и сохранить его в любое место. Например в \ename{path/to/HybMesh/hybmesh.py}.

Для запуска скрипта построения сетки следует перейти в папку, где находится нужный скрипт \ename{script.py},
убедится, что анаконда работает в нужной среде (то есть \ename{conda activate py27} был вызван),
и запустить
\begin{shelloutput}
> python /path/to/HybMesh/hybmesh.py script.py
\end{shelloutput}
