\section{Сеточное решение уравнения Пуассона}
\subsection{Постановка задачи}

Будем рассматривать многомерное дифференциальное уравнение в области $\Omega$:
\begin{equation}
    \label{eq:poissonnd_lam}
    -\nabla\cdot\left(\lambda(\vec x)\nabla u\right) = f(\vec x), \quad \vec x \in \Omega
\end{equation}

Оператор в левой части (оператор Лапласа)
описывает физический процесс диффузии c коэффициентом диффузии $\lambda$.
Это уравнение (с нулевой правой частью) используют
в частности для расчёта распределения температуры в однородном твёрдом теле.
В этом случае коэффиент $\lambda$ называют коэффициентом теплопроводности,
а за счёт ненулевой $f$ можно задавать дополнительные
внутренние источники тепла.

В простом случае постоянного коэффициент диффузии ($\lambda = \const$) его
можно вынести из под дивергеции и отнести в правую часть. Тогда уравнение упростится до однородного вида:
\begin{equation}
    \label{eq:poissonnd}
    -\nabla^2 u = f(\vec x), \quad \vec x \in \Omega
\end{equation}

Далее на границе области расчёта $\partial \Omega$
рассмотрим несколько типов граничных условий.

\subsubsection{Граничные условия первого рода}
Также известны как граничные условия Дирихле.
На границе $\partial \Omega_{I}$ задано точное значение искомой функции $u^\Gamma$:
\begin{equation}
\label{eq:poissonnd_bc}
u = u^\Gamma(\vec x), \quad \vec x \in \partial\Omega_I
\end{equation}

В аналогии задачи теплопроводности это условие
можно трактовать как условие заданной на стенке температуры.

\subsubsection{Граничные условия второго рода}
Также известны как граничные условия Неймана.
На границе $\partial \Omega_{II}$ задано значение нормальной производной искомой функции $q$:
\begin{equation}
\label{eq:poissonnd_bc2}
-\lambda(\vec x)\dfr{u}{n} = q(\vec x), \quad \vec x \in \partial\Omega_{II}
\end{equation}

В аналогии задачи теплопроводности это условие можно трактовать как условие заданного на стенке теплового потока.

\subsubsection{Граничные условия третьего рода}
Также известны как граничные условия Робэна.
На границе $\partial \Omega_{III}$ задано линейное соотношение значений функции и нормальной производной:
\begin{equation}
\label{eq:poissonnd_bc3}
-\lambda(\vec x)\dfr{u}{n} = \alpha(\vec x) u + \beta(\vec x), \quad \vec x \in \partial\Omega_{III}.
\end{equation}

При постановке задачи теплопроводности это условие часто записывают в виде
$$
-\lambda(\vec x)\dfr{u}{n} = \alpha(\vec x)\left(u - u^0\right).
$$
где известное значение $u^0$ называют температурой окружающей стреды.
Такое условие называют условием конвективной теплопроводности или условием Ньютона--Рихмана.
Для приведения этого условия к исходному виду \cref{eq:poissonnd_bc3} достаточно положить $\beta = -\alpha u^0$.

\subsubsubsection{Об универсальности условий третьего рода}
Условия \cref{eq:poissonnd_bc,eq:poissonnd_bc2}
можно свести к условиям \cref{eq:poissonnd_bc3}
при правильном подборе коэффициентов $\alpha$ и $\beta$.
Так для условий второго рода нужно положить $\alpha=0$, $\beta=q$.
А для условий первого рода: $\alpha=\eps^{-1}$, $\beta = -\alpha u^\Gamma$,
где $\eps\to 0$ -- некоторое очень малое положительное число.

\subsubsection{Периодические граничные условия}
Необходимость в таких условиях
возникает при расчёте физических процессов
около периодических структур: решёток, лопастей, рядов скважин, оребрения нагревателя и т.п.
В этом случае из исходную большую область расчёта
представляют как бесконечную последовательность однотипных ячеек периодичности, 
в каждой из которых решения полностью идентичны (в более сложных вариантах -- сдвинуты на константу).

\begin{figure}[h!]
\centering
\includegraphics[width=0.4\linewidth]{periodic_bc.pdf}
\caption{Ячейка периодичности в задаче обтекания бесконечной решётки}
\label{fig:periodic_bc}
\end{figure}
На \figref{fig:periodic_bc} представлен пример области с выделенной ячейкой периодичности $\overline\Omega$ (незатенённая область).
Изолинии можно трактовать как изотермы решения задачи о нестационарном обтекании решётки нагревателя
(поле температур в этом случае описывается более сложным уравнением, чем \cref{eq:poissonnd_lam}
и приведено тут только для иллюстрации периодичности).

Пара периодических границ обозначена через $\partial\overline\Omega_P$ и $\partial\overline\Omega_P'$.
Пусть эти границы топологически экваивалентны, то есть для любой точки $\vec x\in\partial\overline\Omega_P$
cуществует взаимноодносзначная точка $\vec x'\in\partial\overline\Omega_P'$.
Для того, чтобы решение за этими границами точно соответствовало решению внутри ячейки периодичности необходимо
задать равенство значений и производных любого порядка:
\begin{equation}
\label{eq:poissonnd_bcp}
\left\{
\begin{array}{l}
    u(\vec x) = u(\vec x'), \\ [10pt]
    \left.\displaystyle\frac{\partial^k u}{\partial n^k}\right|_{\vec x} = -\left.\displaystyle\frac{\partial^k u}{\partial n^k}\right|_{\vec x'}
\end{array}
\right.
\vec x\in\partial\overline\Omega_P, \quad \vec x'\in\partial\overline\Omega_P', \quad \forall k.
\end{equation}
Здесь под $n$ подразумевается внешняя к ячейке периодичности нормаль, поэтому в правой части
условия для производных стоит минус.



\subsection{Метод конечных разностей}
Рассмотрим задачу \cref{eq:poissonnd,eq:poissonnd_bc} в упрощённой одномерной постановке:
\begin{equation}
    \label{eq:poisson1d}
    -\ddfrq{u}{x} = f(x)
\end{equation}
в области $x\in[a,b]$ с граничными условиями первого рода
\begin{equation}
	\label{eq:poisson1d_bc}
	\begin{cases}
        u(a)=u_a,\\[5pt]
        u(b)=u_b.\\
	\end{cases}
\end{equation}

Необходимо:
\begin{itemize}
\item 
	Запрограммировать расчётную схему для численного решения этого уравнения методом конечных разностей
	на сетке с постоянным шагом,
\item
	С помощью вычислительных экспериментов подтвердить порядок аппроксимации расчётной схемы.
\end{itemize}

\subsubsection{Метод решения}

\subsubsubsection{Нахождение численного решения}

В области решения $[a,b]$ введём равномерную сетку из $N$ ячеек.
Шаг сетки будет равен $h=(b-a)/N$.
Узлы сетки запишем в виде сеточного вектора $\{x_i\}$ длины $N+1$, где $i=\overline{0,N}$.
Определим сеточный вектор $\{u_i\}$ неизвестных, элементы которого определяют значение искомого численного решения в $i$-ом узле сетки. 

Разностная схема второго порядка для уравнения \eqref{eq:poisson1d} имеет вид
\begin{equation}
    \label{eq:poisson1d_fdm}
    \frac{-u_{i-1} + 2u_{i} - u_{i+1}}{h^2} = f_i, \qquad i=\overline{1,N-1}.
\end{equation}
Здесь $\{f_i\}$ -- известный сеточный вектор, определяемый через известную
аналитическую функцию $f(x)$ в правой части уравнения \eqref{eq:poisson1d} как
\begin{equation}
    \label{eq:poisson1d_fdm2}
    f_i = f(x_i).
\end{equation}

Аппроксимация граничных условий \eqref{eq:poisson1d_bc} первого рода даёт дополнительные 
сеточные уравнения для граничных узлов
\begin{equation}
    \label{eq:poisson1d_fdm_bc}
    \begin{array}{ll}
        u_0 = u_a,\\
        u_N = u_b
    \end{array}
\end{equation}

Линейные уравнения \eqref{eq:poisson1d_fdm}, \eqref{eq:poisson1d_fdm_bc}
составляют систему вида

\begin{equation*}
    \sum_{j=0}^{N} A_{ij}\,u_j = b_i, \qquad i=\overline{0,N}
\end{equation*}
с матричными коэффициентами
\begin{equation}
    \label{eq:poisson1d_fdm_lhs}
    A_{ij} = \begin{cases}
        1,      &\quad i=0, \, j=0; \\
        2/h^2,  &\quad i=\overline{1,N-1}, \, j=i;\\
        -1/h^2, &\quad i=\overline{1,N-1}, \, j=i-1;\\
        -1/h^2, &\quad i=\overline{1,N-1}, \, j=i+1;\\
        1,      &\quad i=N, \, j=N; \\
        0,      &\quad \text{иначе}.
    \end{cases}
\end{equation}
и правой частью
\begin{equation}
    \label{eq:poisson1d_fdm_rhs}
    b_i = \begin{cases}
        u_a,   &\quad i=0;\\
        u_b,   &\quad i=N;\\
        f_i,   &\quad i=\overline{1,N-1}.
    \end{cases}
\end{equation}
Искомый вектор находится путём решения этой системы.

\subsubsubsection{Практическое определения порядка аппроксимации}
\label{sec:compute-appr}

Порядок аппрокцимации показывает скорость
приближения численного решения к точному с уменьшением сетки.
Поэтому для подтверждения порядка необходимо
\begin{itemize}
\item Знать точное решение,
\item Уметь вычислять функционал (норму, $||\cdot||$), характеризующий отклонение точного решения от численного,
\item Сделать несколько расчётов на сетках с разной $N$  и заполнить таблицу $||\{u_i - u^e(x_i)\}||(N)$,
\item На основе этой таблицы построить график в логарифмических осях и по углу наклона кривой сделать вывод о порядке аппроксимации.
\end{itemize}

Выберем произвольную функцию $u^e$ (достаточно сильно изменяющуюся на целевом отрезке $[a,b]$).

Далее путём прямого вычисления определим параметры задачи $f$, $u_a$, $u_b$ такие,
для которых функция $u^e$ является точным решением задачи \eqref{eq:poisson1d}, \eqref{eq:poisson1d_bc}.

Зададимся числом разбиений $N$ и решим задачу для выбранным параметров.
В результате определим сеточный вектор численного решения $\{u_i\}$.

В качестве нормы выберем стандартное отклонение. В интегральном виде для многомерной функции $y(\vec x)$
в области $\vec x\in D$ оно имеет вид
\begin{equation}
    \label{eq:norm2_common}
    ||y(\vec x)||_2 = \sqrt{\frac{1}{|D|}\int_{D} y(\vec x)^2 \, d\vec x}.
\end{equation}
Упрощая до одномерного случая
\begin{equation*}
    ||y(x)||_2 = \sqrt{\frac{1}{b-a}\int_{a}^{b} y(x)^2 \, dx}.
\end{equation*}

Вычислим этот интеграл численно на введённой ранее равномерной сетке $\{x_i\}$:
\begin{equation*}
    ||\{y_i\}||_2 = \sqrt{\frac{1}{b-a}\sum_{i=0}^{N} w_i y_i^2},
\end{equation*}
где $\{w_i\}$ -- вес (или "площадь влияния") $i$-ого узла:
\begin{equation*}
    w_i = \begin{cases}
        h/2, &\quad i=0, N;\\
        h, &\quad i=\overline{1,N-1},
    \end{cases}
\end{equation*}
такая что
\begin{equation*}
    \sum_{i=0}^{N} w_i = b-a.
\end{equation*}

Окончательно среднеквадратичная норма отклонения численного решения от точного запишется в виде
\begin{equation}
    \label{eq:poisson1d_fdm_norm}
    ||\{u_i - u^e(x_i)\}||_2 = \sqrt{\frac{1}{b-a}\sum_{i=0}^{N} w_i \left(u_i - u^e_i\right)^2}.
\end{equation}

Пример программы, реализующей этот алгоритм смотри в \secref{sec:poisson1d_prog}.
