\subsection{Лекция 16 (21.02.26) Задача о течении в каверне. SIMPLE + МКР}
Тест с задачей о течении в каверне описан в \secref{sec:prog_cavity_fdm}.
\begin{enumerate}
\item 
Подобрать оптимальные параметры алгоритма SIMPLE $\alpha_u, \alpha_p$ для задачи в каверне,
при которых сходимость происходит за наименьшее число итераций.
Для этого лучше понизить пороговый $\eps=10^{-3}$.
Увеличить разбиение и отметить, как величина шага по простравнству влияет на количество
требуемых итераций.
Для ускорения параметрических расчётов лучше собирать программу в ``релизной''
версии и убрать сохранение в vtk внутри каждой итерации.
Количество итераций, требуемых для сходимости при различных параметров, занести в таблицу.

\item
Нарисовать поле невязок $r_u, r_v$ в динамике по каждой итерации. Отметить в каком из уравнений и в каких местах области расчёта
наблюдаются наибольшие проблемы со сходимостью.

\item Расчитать и нарисовать поля завихрённости $\omega = \dsfr{v}{x} - \dsfr{u}{y}$ и линии тока $\psi$.
Для определения последнего необходимо решить уравнение
\begin{equation*}
-\nabla^2 \psi = \omega, \quad \left.\psi\right|_\Gamma = 0.
\end{equation*}

\item Посчитать суммарный коэффициент трения
\begin{equation*}
\tau_w = -\frac{1}{\Ren}\arint{\dfr{u_\tau}{n}}{\Gamma}{\vec s}.
\end{equation*}
используя касательную скорость ${u_\tau}$ внешнюю нормаль $n$ к границе.
Нарисовать сходимость функционалов $\tau_w$ и $\max|\omega|$ с продвижением по итерациям.

\end{enumerate}

\paragraph{Рекомендации к программированию}
\begin{itemize}
\item п.2: Обратить внимание, что невязка $r_u$ задана на ``красной'' сетке. При этом сохранение на этой сетке делается
через объект \cvar{_writer_u}. Невязка $r_v$ задается на ``синей'' сетке с объектом сохранения \cvar{_writer_v}.
Чтобы активировать сохранения на расчётных сетках, необходимо поставить флаг
\cvar{save_exact_fields} при инициализации сохранения \cvar{initialize_saver}.

\item п.3 Исходя из определения завихренности, легко видеть
что аппроксимировать вторым порядком точности её проще всего на основной сетке $ij$ (\cvar{grid_}).
Для внутренних узлов можно записать:
\begin{equation*}
    \label{eq:omega_approx}
    \omega_{i,j} = \frac{v_{i+\tfrac12,j} - v_{i-\tfrac12,j}}{h_x}
                 - \frac{u_{i, j+\tfrac12} - u_{i, j-\tfrac12}}{h_y}.
\end{equation*}
Для граничных узлов возможно (при известных значениях скорости на границах) использовать направленные разности.
Например, для $i=0$:
\begin{equation*}
    \label{eq:omega_approx_bc}
    \omega_{0,j} = \frac{v_{\tfrac12,j} - v_{0,j}}{h_x/2}
                 - \frac{u_{i, j+\tfrac12} - u_{i, j-\tfrac12}}{h_y}.
\end{equation*}

Получив сеточный вектор для завихрённости ${\omega}$
можно записать разностную схему для 
определения функции тока во внутренних узлах сетки:
\begin{equation*}
    \label{eq:psi_slae}
    \frac{-\psi_{i-1,j} + 2\psi_{i,j} - \psi_{i+1,j}}{h_x^2} +
    \frac{-\psi_{i,j-1} + 2\psi_{i,j} - \psi_{i,j+1}}{h_y^2} 
    = \omega_{i,j}.
\end{equation*}
Расчёт полей $\omega$, $\psi$ следует проводить только в момент сохранения (через объёкт \cvar{write_all_}).

\item п.4
Для определения производной
$\ddfr{u_\tau}{n}$ на границе следует воспользоваться известным граничным условием для скорости и учесть направление внешней нормали.
Например, на отрезке $(i=0, j=[0, 1])$ такая производная определится как
\begin{equation*}
\dfr{u_\tau}{n} \approx \frac12\left(
    \frac{v_{left} - v_{\tfrac12,0}}{h_x/2} +\frac{v_{left} - v_{\tfrac12,1}}{h_x/2}\right), \quad v_{left} = 0.
\end{equation*}

\end{itemize}
