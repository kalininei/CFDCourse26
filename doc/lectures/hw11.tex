\subsection{Лекция 11 (22.11.25) МКЭ решение уравнения переноса. Метод ограничения потока FEM-TVD }
\label{sec:fem_tvd}
В тестах \cvar{[transport_fem]} \cvar{[transport1_fem_tvd]} из файла \ename{transport_fem_test.cpp}
реализовано решение одномерного уравнения переноса
c использованием явных FEM-TVD схем: алгебраическая схема \cref{eq:fem_tvd_algebraic_rij},
схема с интерполяцией в содержащем элементе \cref{eq:fem_tvd_rij_nd,eq:fem_tvd_interpolation_eb},
схема с восстановлением фиктивного значения из градиента
\cref{eq:fem_tvd_rij_nd,eq:fem_tvd_interpolation_gradient}.

По итогу работы каждый тест пишет нестационарное решение в файл
\ename{transport_fem.vtk.series} который можно открыть в Paraview
по аналогии с \secref{sec:tvd_fdm}.

\begin{enumerate}
\item
Реализовать
схему с интерполяцией в инцидентном элементе \cref{eq:fem_tvd_interpolation_ea}
и
схему с использованием противопоточного градиента
\cref{eq:fem_tvd_interpolation_upwind_gradient}.

\item
Отрисовать нестационарные одномерные решения для всех реализованных схем.
Провести анализ сеточной сходимости, зафиксировав шаг по времени $\dt = 10^{-4}$.

\item
Сделать рефакторинг кода, добавив общий абстрактный родительский класс для пар родственных методов 
\cref{eq:fem_tvd_interpolation_ea,eq:fem_tvd_interpolation_ea}
и
\cref{eq:fem_tvd_interpolation_gradient,eq:fem_tvd_interpolation_upwind_gradient}.

\item
Решить двумерную задачу в единичном квадрате $\vec x \in [0, 1]\times[0, 1]$ на треугольной сетке.
В качестве поля скорости использовать $\vec U = (-\pi (y - 0.5), \pi (x - 0.5))$,
Решение в начальный момент времени содержит конус и цилиндр с вырезом (\figref{fig:fem_tvd_test}):
\begin{align*}
\text{Конус:}   & \quad \vec x_0 = (0.5, 0.25), \quad u(r < 1) = 1 - r, \\
\text{Цилиндр:} & \quad \vec x_0 = (0.5, 0.75), \quad u(r < 1) = \begin{cases}
1, \quad & |x - x_0| \geq 0.025 \quad \text{или} \quad y \geq 0.85,\\
0, \quad & \text{иначе}.
\end{cases}
\end{align*}
где для каждой фигуры нормированное расстояние вычисляется как
\begin{equation*}
r = \frac{1}{r_0}\sqrt{(x-x_0)^2 + (y-y_0)^2}, \qquad r_0 = 0.15.
\end{equation*}

\begin{figure}[h!]
\centering
\includegraphics[width=0.4\linewidth]{fem_tvd_test.png}
\caption{Начальное решение u(x, y, 0) на конечноэлементной сетке}
\label{fig:fem_tvd_test}
\end{figure}

Проиллюстрировать решение для всех запрограммированных схем.
\end{enumerate}

\paragraph{Рекомендации к программированию}
\begin{itemize}
\item
Реализацию схемы \cref{eq:fem_tvd_interpolation_ea} 
для п.1 следует осуществлять на основе класса \cvar{ElementBTvd},
в котором нужно реализовать другую схему определения 
интерполяционного элемента \cvar{icell}:
вместо использования геомерического поисковика $CellFinder$
нужно перебрать все элементы, инцидентные узлу $i$:
\begin{cppcode}
worker.grid().tab_point_cell(edge.i)
\end{cppcode}
и выбрать тот, чей центр (\cvar{grid.cell_center()}) ближе к фиктивной точке \cvar{p_upwind}.

\item
Реализацию схемы \cref{eq:fem_tvd_interpolation_upwind_gradient} 
для п.1 следует осуществлять на основе класса \cvar{GradientTvd},
в котором нужно модифицировать 
вес $r^*_{ijk}$, вычисленный по формуле \cref{eq:fem_tvd_gradient_rijk} (\cvar{w} в коде),
согласно формуле \cref{eq:fem_tvd_upwind_gradient_rijk}.

\item
Для программироования двумерного теста из п.4 нужно использовать класс-построитель треугольных конечных элементов
\cvar{FemLinearTriangle} из предыдущего домашнего задания \secref{sec:fct_fem},
модифицировав список базисов, для которых ставятся граничные условия первого рода: теперь в этом 
список нужно положить все граничные узлы.
Для формулирования двумерного решения 
нужно создать свой класс \cvar{Solution2D} реализующий интерфейс \cvar{ISolution}.
Точное решение этой задачи есть начальное решение, совершающее один оборот против часовой стрелки вокруг центра области за время 2.

\item
Для выбора шага по времени для иллюстроации решения двумерной задачи из п.4 следует
иметь ввиду условие устойчивости
\cref{eq:fem_tvd_tau_condition} для явной схемы ($\theta = 0$).
Нужно выбрать такой $\dt$, который бы гарантировал устойчивость всех расчётных схем.



\end{itemize}
