\subsection{Методы решения систем уравнений}
Будем рассматривать систему 
\begin{equation}
\label{eq:sae_generic}
\mat A u = f.
\end{equation}

\subsubsection{Простые итерационные алгоритмы}
\label{sec:sae_simple}
\subsubsubsection{Метод Якоби}
\label{sec:sae_jacobi}
TODO
\subsubsubsection{Метод Зейделя}
\label{sec:sae_seidel}
TODO
\subsubsubsection{Метод последовательных верхних релаксаций SOR}
\label{sec:sae_sor}
TODO
\subsubsection{Метод коррекции поправки}
\label{sec:it_def_corr}
Пусть $u^n$ -- некоторое приближённое решение системы \cref{eq:sae_generic} на итерационном слое.
Введём невязки как
\begin{equation}
\label{eq:sae_rn}
r^n = f - \left(\mat A u\right)^n
\end{equation}
и поправку для перехода на следующий итерационный слой:
\begin{equation*}
\triangle u^n = u^{n+1} - u^{n}.
\end{equation*}
Обобщённый процесс с предобуславливателем $\mat B$ для исходной системы можно записать в виде
\begin{equation}
\label{eq:sae_un_problem}
\mat B \triangle u^n = r^n
\end{equation}

Этот процесс является согласованным: если в качестве $u^n$ взять точное решение системы \cref{eq:sae_generic},
то $r^n = 0$, и следовательно, $\triangle u^n = 0$.

Алгоритм решения исходной системы будет иметь следующий вид:
\begin{enumerate}
\item задать начальное приближение $u^n$ при $n=0$
\item вычислить невязку $r^n$ из \cref{eq:sae_rn}
\item если норма невязки $||r^n|| < \epsilon$, то завершить итерации
\item решив систему \cref{eq:sae_un_problem} получить поправку $\triangle u^n = \mat B^{-1} r^n$.
\item найти следующее приближение $u^{n+1} = u^{n} + \triangle u^{n}$
\item увеличить счётчик $n = n + 1$ и перейти на шаг 2.
\end{enumerate}

Конкретный итерационный метод целиком определяется выбором предобуславливаетеля.
В качестве предобуславливателя обычно выбирают матрицу, похожую на $\mat A$, но при этом легкообратимую.
Некоторые примеры:
\begin{itemize}
\item если выбрать константу $B = 1/\gamma$, то получим
простой итерационный процесс Ричардсона с шагом $\gamma$.
\item диагональная часть исходного оператора $B = \{a_{ii}\}$ приводит к методу Якоби \secref{sec:sae_jacobi}
\item нижняя треугольная часть исходной матрицы $B = \{a_{ij}\}, j\leq i$ даёт метод Зейделя \secref{sec:sae_seidel}.
\end{itemize}

Если в качестве предобуславливателя взять изходную матрицу $\mat A$, 
то итерационный процесс сойдётся за одну итерацию. Действительно из \cref{eq:sae_rn,eq:sae_un_problem}:
\begin{equation*}
u^{n+1} = u^{n} + A^{-1}(f - A u^n) = A^{-1} f
\end{equation*}

Это метод требует прямого обращения только предобуславливаетеля $\mat B$, но не 
исходного оператора $\mat A$, поэтому он может быть применён в том числе и для решения нелинейных систем.

Для обращения самого предобуславливателя в свою очередь так же может
быть использован итерационный алгоритм. Этот итерационный процесс будет являться
внутриннем, поэтому не будет требовать глубокой сходимости.
Зачастую для обращения предобуславливателя достаточно сделать 2-3 поверхностные итерации
одним из простых методов из \secref{sec:sae_simple}.
