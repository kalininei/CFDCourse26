\subsection{Лекция 14 (13.12.25) Решение нелинейного уравнения Баклея-Леверетта}
\label{sec:fvm_bl}
В тесте \cvar{[bl-explicit]} из файла \ename{bl_fvm_test.cpp}
реализовано решение одномерного уравнения Баклея-Леверетта вида
\begin{align*}
&\dfr{s}{t} + \vec U \cdot \nabla f(s) = 0, \\
&f(s) = \frac{s^2}{s^2 + (1-s)^2}, \\
&s(x=0, t) = 1.0, \\
&s(x, t=0) = 0.0.
\end{align*}
где единичная скорость $\vec U$ направлена вдоль оси $x$: $U_x = 1.0$.
Использовалась схема против потока первого порядка точности.
\begin{enumerate}
\item
Исследовать устойчивость upwind-схемы. Подобрать максимальное число Куранта $C=U h /\dt$, при котором решение остаётся монотонным.
Проверить вывод на разных разбиениях.
\item
Решить задачу Римана аналитически и вписать найденное решение в метод \cvar{exact_solution}.
Нарисовать график сеточной сходимости для схемы upwind, зафиксировав малое $\dt$.
\item
Решить задачу Баклея-Леверетта с использованием схемы Годунова. Нарисовать график сеточной сходимости.
%\item
%Использовать схему с ограничением потока MUSCL и ограничителем Барта-Йесперсона. Нарисовать сеточную сходимость.
%\item
%Использовать схему TVD схему с minmod ограничителем. Нарисовать сеточную сходимость.
\end{enumerate}

\paragraph{Рекомендации к программированию}
\begin{figure}[h!]
    \centering
    \includegraphics[width=0.5\linewidth]{bucklev.pdf}
    \caption{Функция Баклея--Леверетта $f(s)$ (синяя линия) выпуклая оболочка (зелёная линия). Значение насыщенности на скачке обозначено $s^*$}
    \label{fig:bucklev}
\end{figure}
\begin{itemize}
\item п.2: Для получения аналитического решения необходимо построить верхнюю выпуклую оболочку
фукнции Баклея-Леверетта (см. рис.\figref{fig:bucklev}).
в точке $s^*$ значение функции $f(s)$ и касательной, проходящей через точку $(0, 0)$:
$
y = f'(s) s
$
должны совпадать. 
Решая соответствующее уравнение получим $s^* = \sfrac{1}{\sqrt 2}$.
На отрезке $[0, s^*]$ (где выпуклая оболочка имеет постоянный наклон) будет наблюдаться скачок  уплотнения.
Скорость движения этого скачка найдем как $a = f'(s^*)$.
Тогда решение задачи Римана будет составной волной вида:
\begin{equation*}
s = \begin{cases}
0,                                      \quad & \sfrac{x}{t} \geq a, \\
\text{решение уравнения } f'(s) = \sfrac{x}{t}, \quad & \sfrac{x}{t} < a
\end{cases}
\end{equation*}
нелинейное уравнение следует решить методом Ньютона на отрезке $s\in[0.5, 1]$.
Требуемую в методе Ньютона производную можно для простоты приблизить симметричной разностной схемой с шагом, много меньшим шага сетки.
\item п.3: Для использования схемы Годунова следует модифицировать 
метод \cvar{riemann_solver}.
Вместо простой противопоточной схемы здесь следует имплементировать решатель задачи Римана на отрезке $[s_R, s_L]$.
По аналогии с п.2 для нахождения $s^*$ следует решить уравнение
\begin{equation*}
P(s^*) = f(s^*) - f'(s^*) (s^* - s_R) - f(s_R) = 0.
\end{equation*}
относительно насыщенности на скачке $s^*$ на отрезке $[0.5, 1]$ (после точки перегиба).
Если на обоих концах этого отрезка функция $P(s)$ имеет одинаковый знак, это значит мы
попали в волну уплотнения или разрежения и следует вернуть противопоточное значение $s_L$.
Если же знаки различаются, то надо решить нелинейное уравнение $P(s) = 0$ и вернуть полученный ответ.
%\item
%Для программирования схемы MUSCL требуется восстановить поведение $s(x)$ в ячейке. Для этого сначала определим градиент решения.
%Он будет равен
%\begin{equation*}
%\sigma_i = \begin{cases}
%\frac{s[i+1] - s[i-1]}/{2h}, \quad & i > 1, i<N-1,\\
%\frac{1 - s[1]}{1.5 h}, \quad & i = 0,\\
%\frac{s[N-2] - 0}{1.5 h}, \quad & i = N-1
%\end{cases}
%\end{equation*}
%Тогда поведение функции в ячейке будет равно $s(x) = s_i + \sigma_i (x - x_i)$.
\end{itemize}
