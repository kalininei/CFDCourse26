\subsection{Программная реализация}
\label{sec:poisson1d_prog}

\clisting{open}{"test/poisson_fdm_test.cpp"}

Тестовая программа для решения одномерного уравнения Пуассона 
реализована в файле \ename{poisson_fdm_solve_test.cpp}.

В качестве аналитической тестовой функции  используется
\begin{equation*}
    u^e = \sin(10 x^2)
\end{equation*}
на отрезке $x\in[0,1]$.

\subsubsection{Функция верхнего уровня}
объявлена как
\clisting{line}{", \"[poisson1-fdm]\")"}
В программе в цикле по набору разбиений \cvar{n_cells}
\clisting{line}{"for (size_t n_cells"}
создаётся решатель для тестовой задачи, использующий заданное число ячеек
\clisting{line}{"worker"}
вычисляется среднеквадратичная норма отклонения численного решения от точного
\clisting{line}{"n2"}
полученное численное решение (вместе с точным) сохраняется в vtk файле\\
\ename{poisson1_n={10,20,...}.vtk}
\clisting{line}{"save_vtk"}
а полученная норма печатается в консоль напротив количества ячеек
\clisting{line}{"cout"}

В результате работы программы в консоли должна отобразиться таблица вида
\begin{shelloutput}
--- [poisson1] ---
10 0.179124
20 0.0407822
50 0.00634718
100 0.00158055
200 0.000394747
500 6.31421e-05
1000 1.57849e-05
\end{shelloutput}
где первый столбец -- это количество ячеек, а второй -- полученная для этого количества ячеек норма.
Нарисовав график этой таблицы в логарифмических осях подтвердим второй порядок аппроксимации (\figref{fig:poisson_convergence}).

\begin{figure}[h]
\centering
\includegraphics[width=0.9\linewidth]{poisson1_appr.png}
\caption{Сходимость с уменьшением разбиения при решении одномерного уравнения Пуассона}
\label{fig:poisson_convergence}
\end{figure}

Открыв один из cохранённых в процессе работы файлов vtk \ename{poisson1_ncells=?.vtk} в paraview
можно посмотреть полученные графики. В файле представлены как точное ``exact'', так и численное решение ``numerical''
(\figref{fig:poisson_graph}).

\begin{figure}[h]
\centering
\includegraphics[width=0.9\linewidth]{poisson1_graph.png}
\caption{Сравнение точного и численного решений уравнения Пуассона}
\label{fig:poisson_graph}
\end{figure}


\subsubsection{Детали реализации}
\clisting{open}{"test/poisson_fdm_test.cpp"}
Основная работа по решению задачи проводится в классе \cvar{TestPoisson1Worker}.

В его конструкторе происходит инициализация сетки (приватного поля класса) на отрезке $[0, 1]$ с заданным разбиением
\cvar{n_cells}:
\clisting{line}{"TestPoisson1Worker"}

В методе \cvar{solve()} производится численное решения задачи и вычисления нормы.
Для этого последовательно
\begin{enumerate}
\item Строится матрица левой части и вектор правой части определяющей системы уравнений.
      Матрицы хранятся в разреженном формате CSR (\secref{sec:csr}), удобном для последовательного чтения.
\item Вызывается решатель СЛАУ. Решение записывается в приватное поле класса \cvar{u}.
\item Вызывается функция вычисления нормы.
\end{enumerate}

\clisting{block}{"double solve()"}

Функции нижнего уровня (используемые в методе \cvar{solve}):
\begin{itemize}
\item
  Сборка левой части СЛАУ. Реализует формулу \eqref{eq:poisson1d_fdm_lhs}.
  Для заполнения матрицы используется формат \cvar{cfd::LodMatrix} (\secref{sec:lodmat}), удобный для непоследовательной записи, который в конце конвертируется CSR.
  \clisting{block}{"approximate_lhs("}
\item
  Сборка правой части СЛАУ. Реализует формулу \eqref{eq:poisson1d_fdm_rhs}.
  \clisting{block}{"approximate_rhs("}
\item
  Вычисление нормы. Реализует формулу \eqref{eq:poisson1d_fdm_norm}.
  \clisting{block}{"compute_norm2"}
\end{itemize}
